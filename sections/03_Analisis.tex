\chapter{Análisis del problema}\label{cap:analisis}

\section{Descripción de la propuesta}
En este capítulo se analizará el problema que pretende resolver el proyecto,
describiendo la situación actual en la gestión de horarios académicos personalizados en la Universidad de Granada y la necesidad de una solución eficiente.
Actualmente, la ETSIIT no dispone de un sistema automatizado que permita a los estudiantes y profesores gestionar sus horarios de forma personalizada e
integrada con servicios de calendario externos. La asignación de horarios se publica en documentos PDF o en distintas páginas web de la universidad, 
lo que dificulta la consulta y la sincronización con otras herramientas de gestión de tiempo.

La solución propuesta consiste en desarrollar un sistema basado en microservicios que permita la suscripción a horarios personalizados y su sincronización con Google Calendar y otras herramientas similares.
A su vez el sistema avisará a los usuarios de cambios en los horarios y permitirá la gestión de preferencias y restricciones de los mismos.

\section{Metodología de desarrollo}
La metodología de desarrollo que se ha seguido es SCRUM.

\section{Artefactos SCRUM}

\subsection{Product Backlog}

\subsection{Sprint Backlogs}

\subsection{Incrementos}

\section{Requisitos del sistema}
Los requisitos de información son...

\subsection{Requisitos funcionales}
Los requisitos funcionales son...

\subsection{Requisitos no funcionales}
Los requisitos no funcionales son...

\subsection{Requisitos de información}

\section{Planificación y presupuesto}

\subsection{Planificación temporal}

\subsection{Presupuesto}
