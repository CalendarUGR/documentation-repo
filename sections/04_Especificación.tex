\chapter{Especificación de requisitos}\label{cap:especificación}

\section{Metodología de desarrollo}
Para la gestión y desarrollo del proyecto, se ha optado por la metodología ágil \hyperlink{scrum}{Scrum}. Esta metodología se caracteriza por su enfoque iterativo e incremental, permitiendo una adaptación flexible a los cambios y una entrega temprana de valor.

\subsection{Roles y Responsabilidades en este Proyecto}

Dada la naturaleza individual de este proyecto, los roles tradicionales de Scrum se han adaptado de la siguiente manera:

\begin{itemize}
    \item \textbf{Equipo de Desarrollo y Scrum Master:} El autor de este TFG ha asumido ambos roles. Esto implica la responsabilidad de llevar a cabo el desarrollo del software, así como de facilitar el proceso Scrum, asegurando que se sigan las prácticas y principios de la metodología. Se ha encargado de la planificación, ejecución y revisión de cada sprint, así como de la identificación y resolución de impedimentos.
    \item \textbf{Product Owner:} El rol de Product Owner ha sido desempeñado tanto por el director del TFG, D. Juan Luis Jiménez Laredo, como por el autor del sistema. En esta función, ambos han sido los responsables de definir la visión del producto, priorizar el Backlog del Producto y asegurar que el desarrollo se alinee con las necesidades y expectativas del proyecto. Los dos participaron activamente en la definición de los requisitos y en la validación de los incrementos de software.
\end{itemize}

\subsection{Proceso Scrum Implementado}

El proceso Scrum se ha implementado siguiendo los siguientes pasos clave:

\begin{itemize}
    \item \textbf{Backlog del Producto:} Se ha definido un Backlog del Producto inicial, compuesto por las funcionalidades y tareas necesarias para completar el TFG.
    \item \textbf{Sprints:} El desarrollo se ha dividido en sprints de duración determinada (la duración específica de los sprints se definirá en el apartado de planificación \ref{cap:planificacion}). Cada sprint ha tenido como objetivo la entrega de un incremento de software funcional y potencialmente entregable.
    \item \textbf{Planificación del Sprint:} Al inicio de cada sprint, se ha llevado a cabo una reunión de planificación en la que, junto con el Product Owner, se han seleccionado los elementos del Backlog del Producto que se abordarían durante el sprint. Se han estimado las tareas y se ha definido el Sprint Backlog.
    \item \textbf{Desarrollo del Sprint:} Durante el sprint, el autor ha trabajado en el desarrollo de las tareas asignadas, siguiendo las prácticas de desarrollo y asegurando la calidad del código.
    \item \textbf{Reunión Diaria (Daily Scrum):} Aunque adaptada a la naturaleza individual del proyecto, se ha realizado una reflexión diaria sobre el progreso, los impedimentos y las tareas a realizar. Esto ha permitido mantener un seguimiento constante del avance.
    \item \textbf{Revisión del Sprint (Sprint Review):} Al finalizar cada sprint, se ha llevado a cabo una revisión del sprint. Dado que el autor es también el equipo de desarrollo, esta revisión ha consistido en una \textbf{introspección personal y un análisis de los resultados del sprint}, evaluando las metas alcanzadas y el incremento de software desarrollado. Se ha realizado una autoevaluación del progreso y la calidad del trabajo.
    \item \textbf{Retrospectiva del Sprint (Sprint Retrospective):} La retrospectiva del sprint se ha realizado en colaboración con el Product Owner (D. Juan Luis Jiménez Laredo). En esta reunión, se ha analizado el sprint finalizado, identificando qué se ha hecho bien, qué se podría mejorar y qué acciones concretas se podrían implementar para el siguiente sprint. Esta colaboración ha permitido obtener una perspectiva externa y valiosa para la mejora continua del proceso.
\end{itemize}

\subsection{Gestión de Tareas y Seguimiento del Progreso}

Para la gestión de las tareas y el seguimiento del progreso del proyecto, se ha utilizado \textbf{GitHub Projects}. Esta herramienta ha permitido:

\begin{itemize}
    \item \textbf{Creación de Tableros por Sprint:} Se han configurado tableros de proyecto en GitHub Projects, utilizando las funcionalidades de \textbf{``Iteraciones''} para representar cada sprint. Esto ha facilitado la visualización del trabajo en curso para cada iteración.
    \item \textbf{Gestión del Estado de las Tareas:} Dentro de cada tablero de sprint, se han creado y gestionado las tareas individuales, asignándoles diferentes estados ( ``Backlog'', ``Todo'', ``In progress'', ``Testing'', ``Done''). Esto ha permitido un seguimiento visual del avance de cada tarea y del sprint en general.
    \item \textbf{Organización y Priorización:} GitHub Projects ha facilitado la organización de las tareas y su priorización dentro de cada sprint, alineándose con el Sprint Backlog definido.
    \item \textbf{Visualización de las tareas en el tiempo:} La herramienta ha permitido visualizar el progreso de las tareas en el tiempo a través de un roadmap, lo que ha facilitado la identificación de posibles retrasos y la toma de decisiones para ajustar el plan si es necesario.
\end{itemize}

\begin{figure}[h!] 
    \centering 
    \includegraphics[width=1\textwidth]{figures/04_github_tablero.png}
    \caption{Tablero del 2º Sprint durante su desarrollo.} % Leyenda de la imagen
    \label{tablero_github} % Etiqueta para referenciar la imagen
\end{figure}

\begin{figure}[h!] 
    \centering 
    \includegraphics[width=1\textwidth]{figures/04_roadmap.png}
    \caption{Segmento del ''Roadmap'' del proyecto.} % Leyenda de la imagen
    \label{roadmap_github} % Etiqueta para referenciar la imagen
\end{figure}

\subsection{Justificación de la Metodología}

La elección de la metodología Scrum se justifica por las siguientes razones:

\begin{itemize}
    \item \textbf{Flexibilidad:} Permite adaptarse a los cambios en los requisitos y a los aprendizajes obtenidos durante el desarrollo. En concreto este sistema dependía en etapas tempranas de desarrollo del posible acceso a datos oficiale de la UGR, sistemas de autenticación internos, datos de matriculaciones, etc. Es por ello que la flexibilidad de Scrum ha sido clave para ajustar el plan a medida que se han ido conociendo más detalles.
    \item \textbf{Entrega Temprana de Valor:} Facilita la entrega de incrementos funcionales de software de forma regular, lo que permite obtener retroalimentación temprana y ajustar el rumbo del proyecto si es necesario.
    \item \textbf{Transparencia:} El uso de herramientas como GitHub Projects y la realización de las reuniones Scrum promueven la transparencia en el progreso del proyecto.
    \item \textbf{Adaptabilidad a un Proyecto Individual:} Aunque tradicionalmente Scrum se aplica a equipos, su estructura iterativa y adaptable se ajusta bien a un proyecto individual como un TFG, permitiendo una organización eficiente del trabajo y una gestión del tiempo efectiva.
\end{itemize}

Es importante destacar que, dada la naturaleza individual del proyecto, se ha realizado una adaptación de los roles y las ceremonias de Scrum para ajustarse a las necesidades y recursos disponibles. Sin embargo, se han mantenido los principios fundamentales de la metodología para asegurar una gestión eficaz del desarrollo.

\section{Personas}

\section{Escenarios}

\section{Historias de usuario}

Una historia de usuario es una explicación general e informal de una función de software escrita desde la perspectiva
del usuario final. Su propósito es articular cómo proporcionará una función de software valor al cliente. \cite{atlassian_user_stories}.

Estas no usan un lenguaje técnico y preciso para definir y acotar los requisitos de un sistema, sino que se enfocan en
el usuario final y en cómo este interactuará con el sistema. Por lo tanto, las historias de usuario son una herramienta
de comunicación entre el equipo de desarrollo y el cliente.

En \hyperlink{scrum}{Scrum} las historias de usuario son una parte fundamental del proceso de desarrollo de software. En
este marco de trabajo, las historias de usuario son utilizadas para definir los requisitos del sistema y son la base para
la planificación y estimación de las tareas a realizar.
\newline\newline
\negritas{¿Por qué son importantes las historias de usuario?}
\begin{itemize}
    \item Centran la atención en el usuario final.
    \item Permiten la colaboración y comunicación entre el equipo de desarrollo y el cliente.
    \item Fomentan soluciones creativas y flexibles.
\end{itemize}
\subsection{Estructura de una historia de usuario}

Las historias de usuario siguen una estructura general simple y clara. 
\newline\newline
\textbf{Como} [tipo de usuario], \textbf{quiero} [realizar una acción], \textbf{para} [obtener un beneficio].

\begin{itemize}
    \item \textbf{Como}: describe el tipo de usuario que está interactuando con el sistema.
    \item \textbf{Quiero}: describe la acción que el usuario desea realizar.
    \item \textbf{Para}: describe el beneficio que el usuario obtendrá al realizar la acción.   
\end{itemize}

Además de esta estructura general, las historias de usuario pueden incluir otros elementos como criterios de aceptación,
prioridad, estimación de esfuerzo, entre otros.

Para se ha definido la siguiente estructura para las historias de usuario:

%% Tabla: Estructura de una historia de usuario
\begin{table}[H]
    \centering
    \begin{tabular}{|p{2cm}|p{4cm}|p{2cm}|p{4cm}|}
        \hline
        \textbf{ID} & Identificador único de la historia de usuario. & \textbf{Nombre} & Nombre de la historia de usuario. \\
        \hline
        \multicolumn{2}{|p{6cm}|}{\textbf{Descripción}} & \multicolumn{2}{p{6cm}|}{Descripción general de la historia de usuario.} \\
        \hline
        \multicolumn{2}{|p{6cm}|}{\textbf{Estimación}} & \multicolumn{2}{p{6cm}|}{Estimación del esfuerzo necesario para completar la historia de usuario. Basado en Planning Poker.} \\
        \hline
        \multicolumn{2}{|p{6cm}|}{\textbf{Prioridad}} & \multicolumn{2}{p{6cm}|}{Acción que el usuario desea realizar. Desde P3 (baja) hasta P0 (alta).} \\
        \hline
        \multicolumn{2}{|p{6cm}|}{\textbf{Criterios de aceptación}} & \multicolumn{2}{p{6cm}|}{Conjunto de condiciones que deben cumplirse para considerar la historia de usuario como completada.} \\
        \hline
    \end{tabular}
    \caption{Estructura de una historia de usuario}
    \label{tab:estructura_historia_usuario}
\end{table}

\subsection{Historias de usuario}

\begin{table}[H]
    \centering
    \begin{tabular}{|p{2cm}|p{4cm}|p{2cm}|p{4cm}|}
        \hline
        \textbf{ID} & HU-1 & \textbf{Nombre} & Iniciar sesión \\
        \hline
        \multicolumn{2}{|p{6cm}|}{\textbf{Descripción}} & \multicolumn{2}{p{6cm}|}{Como usuario he de poder iniciar sesión en el sistema.} \\
        \hline
        \multicolumn{2}{|p{6cm}|}{\textbf{Estimación}} & \multicolumn{2}{p{6cm}|}{3} \\
        \hline
        \multicolumn{2}{|p{6cm}|}{\textbf{Prioridad}} & \multicolumn{2}{p{6cm}|}{P0} \\
        \hline
        \multicolumn{2}{|p{6cm}|}{\textbf{Criterios de aceptación}} & \multicolumn{2}{p{6cm}|}{
            \begin{itemize}
                \item Para poder iniciar sesión ha de insertar su correo y contraseña.
                \item Sólo se puede iniciar sesión con correos de la UGR.
            \end{itemize}
        } \\
        \hline
    \end{tabular}
    \caption{Historia de usuario HU-1}
    \label{tab:hu_1}
\end{table}

\begin{table}[H]
    \centering
    \begin{tabular}{|p{2cm}|p{4cm}|p{2cm}|p{4cm}|}
        \hline
        \textbf{ID} & HU-2 & \textbf{Nombre} & Registrarse \\
        \hline
        \multicolumn{2}{|p{6cm}|}{\textbf{Descripción}} & \multicolumn{2}{p{6cm}|}{Como usuario he de poder registrarme en el sistema.} \\
        \hline
        \multicolumn{2}{|p{6cm}|}{\textbf{Estimación}} & \multicolumn{2}{p{6cm}|}{5} \\
        \hline
        \multicolumn{2}{|p{6cm}|}{\textbf{Prioridad}} & \multicolumn{2}{p{6cm}|}{P0} \\
        \hline
        \multicolumn{2}{|p{6cm}|}{\textbf{Criterios de aceptación}} & \multicolumn{2}{p{6cm}|}{
            \begin{itemize}
                \item El alumno sólo se puede registrar con su correo institucional de la UGR.
                \item El alumno debe insertar nickname, correo y contraseña.
                \item La contraseña del alumno ha de ser mayor o igual a 9 caracteres, conteniendo esta una mayúscula y un número como mínimo.
                \item El registro se ha de completar mediante un link mandado por mail.
            \end{itemize}
        } \\
        \hline
    \end{tabular}
    \caption{Historia de usuario HU-2}
    \label{tab:hu_2}
\end{table}

\begin{table}[H]
    \centering
    \begin{tabular}{|p{2cm}|p{4cm}|p{2cm}|p{4cm}|}
        \hline
        \textbf{ID} & HU-3 & \textbf{Nombre} & Modificar nickname \\
        \hline
        \multicolumn{2}{|p{6cm}|}{\textbf{Descripción}} & \multicolumn{2}{p{6cm}|}{Como usuario puedo modificar mi nickname.} \\
        \hline
        \multicolumn{2}{|p{6cm}|}{\textbf{Estimación}} & \multicolumn{2}{p{6cm}|}{2} \\
        \hline
        \multicolumn{2}{|p{6cm}|}{\textbf{Prioridad}} & \multicolumn{2}{p{6cm}|}{P2} \\
        \hline
        \multicolumn{2}{|p{6cm}|}{\textbf{Criterios de aceptación}} & \multicolumn{2}{p{6cm}|}{
            \begin{itemize}
                \item El alumno no puede cambiar su nickname a otro que exista.
                \item El alumno no puede modificar su correo electrónico.
            \end{itemize}
        } \\
        \hline
    \end{tabular}
    \caption{Historia de usuario HU-3}
    \label{tab:hu_3}
\end{table}

\begin{table}[H]
    \centering
    \begin{tabular}{|p{2cm}|p{4cm}|p{2cm}|p{4cm}|}
        \hline
        \textbf{ID} & HU-4 & \textbf{Nombre} & Modificar contraseña \\
        \hline
        \multicolumn{2}{|p{6cm}|}{\textbf{Descripción}} & \multicolumn{2}{p{6cm}|}{Como usuario he de poder modificar la contraseña de acceso.} \\
        \hline
        \multicolumn{2}{|p{6cm}|}{\textbf{Estimación}} & \multicolumn{2}{p{6cm}|}{3} \\
        \hline
        \multicolumn{2}{|p{6cm}|}{\textbf{Prioridad}} & \multicolumn{2}{p{6cm}|}{P1} \\
        \hline
        \multicolumn{2}{|p{6cm}|}{\textbf{Criterios de aceptación}} & \multicolumn{2}{p{6cm}|}{
            \begin{itemize}
                \item Para poder modificar la contraseña ha de insertar la contraseña anterior.
                \item Se ha de insertar la nueva contraseña 2 veces, siendo esta  mayor o igual a 9 caracteres, y conteniendo una mayúscula y un número como mínimo.
            \end{itemize}
        } \\
        \hline
    \end{tabular}
    \caption{Historia de usuario HU-4}
    \label{tab:hu_4}
\end{table}

\begin{table}[H]
    \centering
    \begin{tabular}{|p{2cm}|p{4cm}|p{2cm}|p{4cm}|}
        \hline
        \textbf{ID} & HU-5 & \textbf{Nombre} & Darse de baja \\
        \hline
        \multicolumn{2}{|p{6cm}|}{\textbf{Descripción}} & \multicolumn{2}{p{6cm}|}{Como usuario
        he de poder darme de baja del sistema.} \\
        \hline
        \multicolumn{2}{|p{6cm}|}{\textbf{Estimación}} & \multicolumn{2}{p{6cm}|}{1} \\
        \hline
        \multicolumn{2}{|p{6cm}|}{\textbf{Prioridad}} & \multicolumn{2}{p{6cm}|}{P2} \\
        \hline
        \multicolumn{2}{|p{6cm}|}{\textbf{Criterios de aceptación}} & \multicolumn{2}{p{6cm}|}{
            \begin{itemize}
                \item Para poder completar la baja ha de escribir su contraseña en un campo de texto.
            \end{itemize}
        } \\
        \hline
    \end{tabular}
    \caption{Historia de usuario HU-5}
    \label{tab:hu_5}
\end{table}

\begin{table}[H]
    \centering
    \begin{tabular}{|p{2cm}|p{4cm}|p{2cm}|p{4cm}|}
        \hline
        \textbf{ID} & HU-6 & \textbf{Nombre} & Cambiar rol \\
        \hline
        \multicolumn{2}{|p{6cm}|}{\textbf{Descripción}} & \multicolumn{2}{p{6cm}|}{Como administrador he de poder actualizar mi rol a profesor, y viceversa.} \\
        \hline
        \multicolumn{2}{|p{6cm}|}{\textbf{Estimación}} & \multicolumn{2}{p{6cm}|}{2} \\
        \hline
        \multicolumn{2}{|p{6cm}|}{\textbf{Prioridad}} & \multicolumn{2}{p{6cm}|}{P1} \\
        \hline
        \multicolumn{2}{|p{6cm}|}{\textbf{Criterios de aceptación}} & \multicolumn{2}{p{6cm}|}{
            \begin{itemize}
                \item Para poder cambiar el rol a profesor he de ser administrador.
                \item Para poder cambiar el rol a profesor he de ser administrador.
            \end{itemize}
        } \\
        \hline
    \end{tabular}
    \caption{Historia de usuario HU-6}
    \label{tab:hu_6}
\end{table}

\begin{table}[H]
    \centering
    \begin{tabular}{|p{2cm}|p{4cm}|p{2cm}|p{4cm}|}
        \hline
        \textbf{ID} & HU-7 & \textbf{Nombre} & Seleccionar grados \\
        \hline
        \multicolumn{2}{|p{6cm}|}{\textbf{Descripción}} & \multicolumn{2}{p{6cm}|}{Como usuario he de poder seleccionar el grado o grados que estoy cursando.} \\
        \hline
        \multicolumn{2}{|p{6cm}|}{\textbf{Estimación}} & \multicolumn{2}{p{6cm}|}{3} \\
        \hline
        \multicolumn{2}{|p{6cm}|}{\textbf{Prioridad}} & \multicolumn{2}{p{6cm}|}{P0} \\
        \hline
        \multicolumn{2}{|p{6cm}|}{\textbf{Criterios de aceptación}} & \multicolumn{2}{p{6cm}|}{
            \begin{itemize}
                \item Se pueden seleccionar un máximo de 4 grados.
            \end{itemize}
        } \\
        \hline
    \end{tabular}
    \caption{Historia de usuario HU-7}
    \label{tab:hu_7}
\end{table}

\begin{table}[H]
    \centering
    \begin{tabular}{|p{2cm}|p{4cm}|p{2cm}|p{4cm}|}
        \hline
        \textbf{ID} & HU-8 & \textbf{Nombre} & Eliminar grado \\
        \hline
        \multicolumn{2}{|p{6cm}|}{\textbf{Descripción}} & \multicolumn{2}{p{6cm}|}{Como usuario he de poder eliminar un grado que ya no esté cursando.} \\
        \hline
        \multicolumn{2}{|p{6cm}|}{\textbf{Estimación}} & \multicolumn{2}{p{6cm}|}{2} \\
        \hline
        \multicolumn{2}{|p{6cm}|}{\textbf{Prioridad}} & \multicolumn{2}{p{6cm}|}{P1} \\
        \hline
        \multicolumn{2}{|p{6cm}|}{\textbf{Criterios de aceptación}} & \multicolumn{2}{p{6cm}|}{
            \begin{itemize}
                \item Si el usuario está suscrito a grupos de asignatura de ese grado, se le recordará que también se revocarán sus suscripciones a estos.
            \end{itemize}
        } \\
        \hline
    \end{tabular}
    \caption{Historia de usuario HU-8}
    \label{tab:hu_8}
\end{table}

\begin{table}[H]
    \centering
    \begin{tabular}{|p{2cm}|p{4cm}|p{2cm}|p{4cm}|}
        \hline
        \textbf{ID} & HU-9 & \textbf{Nombre} & Suscribirse a grupos de asignatura \\
        \hline
        \multicolumn{2}{|p{6cm}|}{\textbf{Descripción}} & \multicolumn{2}{p{6cm}|}{Como usuario he de poder suscribirme a los grupos de asignaturas a las que quiero hacer seguimiento.} \\
        \hline
        \multicolumn{2}{|p{6cm}|}{\textbf{Estimación}} & \multicolumn{2}{p{6cm}|}{4} \\
        \hline
        \multicolumn{2}{|p{6cm}|}{\textbf{Prioridad}} & \multicolumn{2}{p{6cm}|}{P0} \\
        \hline
        \multicolumn{2}{|p{6cm}|}{\textbf{Criterios de aceptación}} & \multicolumn{2}{p{6cm}|}{
            \begin{itemize}
                \item Para hacer seguimiento a un grupo en concreto, el usuario deberá estar cursando el grado al que pertenece.
            \end{itemize}
        } \\
        \hline
    \end{tabular}
    \caption{Historia de usuario HU-9}
    \label{tab:hu_9}
\end{table}

\begin{table}[H]
    \centering
    \begin{tabular}{|p{2cm}|p{4cm}|p{2cm}|p{4cm}|}
        \hline
        \textbf{ID} & HU-10 & \textbf{Nombre} & Revocar suscripción a grupo de asignatura \\
        \hline
        \multicolumn{2}{|p{6cm}|}{\textbf{Descripción}} & \multicolumn{2}{p{6cm}|}{Como usuario he de poder revocar una suscripción a un grupo de asignatura.} \\
        \hline
        \multicolumn{2}{|p{6cm}|}{\textbf{Estimación}} & \multicolumn{2}{p{6cm}|}{2} \\
        \hline
        \multicolumn{2}{|p{6cm}|}{\textbf{Prioridad}} & \multicolumn{2}{p{6cm}|}{P1} \\
        \hline
        \multicolumn{2}{|p{6cm}|}{\textbf{Criterios de aceptación}} & \multicolumn{2}{p{6cm}|}{
            \begin{itemize}
                \item El usuario sólo puede revocar suscripciones de grupos a los que está suscrito.
            \end{itemize}
        } \\
        \hline
    \end{tabular}
    \caption{Historia de usuario HU-10}
    \label{tab:hu_10}
\end{table}

\begin{table}[H]
    \centering
    \begin{tabular}{|p{2cm}|p{4cm}|p{2cm}|p{4cm}|}
        \hline
        \textbf{ID} & HU-11 & \textbf{Nombre} & Ver horario de grupos de asignatura \\
        \hline
        \multicolumn{2}{|p{6cm}|}{\textbf{Descripción}} & \multicolumn{2}{p{6cm}|}{Como usuario he de poder obtener la información de mi horario personalizado conforme a las suscripciones.} \\
        \hline
        \multicolumn{2}{|p{6cm}|}{\textbf{Estimación}} & \multicolumn{2}{p{6cm}|}{5} \\
        \hline
        \multicolumn{2}{|p{6cm}|}{\textbf{Prioridad}} & \multicolumn{2}{p{6cm}|}{P0} \\
        \hline
        \multicolumn{2}{|p{6cm}|}{\textbf{Criterios de aceptación}} & \multicolumn{2}{p{6cm}|}{
            \begin{itemize}
                \item El horario de cada clase debe mostrar la asignatura, grupo, hora de inicio y fin, profesores del grupo, y aula.
                \item Las clases sólo deben mostrarse en el rango de fechas en las que se imparten.
            \end{itemize}
        } \\
        \hline
    \end{tabular}
    \caption{Historia de usuario HU-11}
    \label{tab:hu_11}
\end{table}

\begin{table}[H]
    \centering
    \begin{tabular}{|p{2cm}|p{4cm}|p{2cm}|p{4cm}|}
        \hline
        \textbf{ID} & HU-12 & \textbf{Nombre} & Crear evento puntual a nivel de grupo de asignatura \\
        \hline
        \multicolumn{2}{|p{6cm}|}{\textbf{Descripción}} & \multicolumn{2}{p{6cm}|}{Como profesor / administrador he de poder crear eventos puntuales a nivel de grupo ( clases de recuperación, extra, charlas …).} \\
        \hline
        \multicolumn{2}{|p{6cm}|}{\textbf{Estimación}} & \multicolumn{2}{p{6cm}|}{3} \\
        \hline
        \multicolumn{2}{|p{6cm}|}{\textbf{Prioridad}} & \multicolumn{2}{p{6cm}|}{P0} \\
        \hline
        \multicolumn{2}{|p{6cm}|}{\textbf{Criterios de aceptación}} & \multicolumn{2}{p{6cm}|}{
            \begin{itemize}
                \item Se debe especificar la fecha, hora de inicio, hora de fin y tipo de evento.
                \item La clase extra no debe coincidir con otra clase existente en horario y aula.
                \item El usuario debe ser un profesor o administrador.
            \end{itemize}
        } \\
        \hline
    \end{tabular}
    \caption{Historia de usuario HU-12}
    \label{tab:hu_12}
\end{table}

\begin{table}[H]
    \centering
    \begin{tabular}{|p{2cm}|p{4cm}|p{2cm}|p{4cm}|}
        \hline
        \textbf{ID} & HU-13 & \textbf{Nombre} & Eliminar evento puntual a nivel de grupo de asignatura \\
        \hline
        \multicolumn{2}{|p{6cm}|}{\textbf{Descripción}} & \multicolumn{2}{p{6cm}|}{Como profesor / administrador he de poder eliminar los eventos que he creado ( clases de recuperación, extra, charlas …).} \\
        \hline
        \multicolumn{2}{|p{6cm}|}{\textbf{Estimación}} & \multicolumn{2}{p{6cm}|}{2} \\
        \hline
        \multicolumn{2}{|p{6cm}|}{\textbf{Prioridad}} & \multicolumn{2}{p{6cm}|}{P1} \\
        \hline
        \multicolumn{2}{|p{6cm}|}{\textbf{Criterios de aceptación}} & \multicolumn{2}{p{6cm}|}{
            \begin{itemize}
                \item Se debe especificar la fecha, hora de inicio, hora de fin y tipo de evento.
                \item La clase extra no debe coincidir con otra clase existente en horario y aula.
                \item El usuario debe ser un profesor o administrador.
            \end{itemize}
        } \\
        \hline
    \end{tabular}
    \caption{Historia de usuario HU-13}
    \label{tab:hu_13}
\end{table}

\begin{table}[H]
    \centering
    \begin{tabular}{|p{2cm}|p{4cm}|p{2cm}|p{4cm}|}
        \hline
        \textbf{ID} & HU-14 & \textbf{Nombre} & Exportar horario a calendario estándar \\
        \hline
        \multicolumn{2}{|p{6cm}|}{\textbf{Descripción}} & \multicolumn{2}{p{6cm}|}{Como usuario he de poder exportar mi horario para usarlo en otros sistemas estándar de calendario.} \\
        \hline
        \multicolumn{2}{|p{6cm}|}{\textbf{Estimación}} & \multicolumn{2}{p{6cm}|}{4} \\
        \hline
        \multicolumn{2}{|p{6cm}|}{\textbf{Prioridad}} & \multicolumn{2}{p{6cm}|}{P0} \\
        \hline
        \multicolumn{2}{|p{6cm}|}{\textbf{Criterios de aceptación}} & \multicolumn{2}{p{6cm}|}{
            \begin{itemize}
                \item El usuario podrá exportar su horario en formatos compatibles con sistemas estándar de calendario (.ics).
                \item La exportación debe incluir todas las asignaturas y eventos del usuario.
                \item Se debe permitir elegir un rango de fechas para la exportación.
                \item El archivo generado debe poder descargarse y ser importable en Google Calendar, Outlook, Apple Calendar, etc.
            \end{itemize}
        } \\
        \hline
    \end{tabular}
    \caption{Historia de usuario HU-14}
    \label{tab:hu_14}
\end{table}

\begin{table}[H]
    \centering
    \begin{tabular}{|p{2cm}|p{4cm}|p{2cm}|p{4cm}|}
        \hline
        \textbf{ID} & HU-15 & \textbf{Nombre} & Sincronizar calendario con Google Calendar \\
        \hline
        \multicolumn{2}{|p{6cm}|}{\textbf{Descripción}} & \multicolumn{2}{p{6cm}|}{Como usuario he de poder sincronizar mi calendario con Google Calendar.} \\
        \hline
        \multicolumn{2}{|p{6cm}|}{\textbf{Estimación}} & \multicolumn{2}{p{6cm}|}{5} \\
        \hline
        \multicolumn{2}{|p{6cm}|}{\textbf{Prioridad}} & \multicolumn{2}{p{6cm}|}{P3} \\
        \hline
        \multicolumn{2}{|p{6cm}|}{\textbf{Criterios de aceptación}} & \multicolumn{2}{p{6cm}|}{
            \begin{itemize}
                \item El usuario podrá vincular su cuenta con Google Calendar mediante OAuth.
                \item Los eventos de su horario deben sincronizarse automáticamente con Google Calendar.
                \item Se deben reflejar en Google Calendar los cambios realizados en el horario del usuario.
                \item El usuario debe poder desactivar la sincronización en cualquier momento.
            \end{itemize}
        } \\
        \hline
    \end{tabular}
    \caption{Historia de usuario HU-15}
    \label{tab:hu_15}
\end{table}

\begin{table}[H]
    \centering
    \begin{tabular}{|p{2cm}|p{4cm}|p{2cm}|p{4cm}|}
        \hline
        \textbf{ID} & HU-16 & \textbf{Nombre} & Ver alertas de clases extra de grupos de asignatura \\
        \hline
        \multicolumn{2}{|p{6cm}|}{\textbf{Descripción}} & \multicolumn{2}{p{6cm}|}{Como alumno he de poder recibir alertas referentes a “clases extra” de mis grupos ( clases de recuperación, extra, charlas …).} \\
        \hline
        \multicolumn{2}{|p{6cm}|}{\textbf{Estimación}} & \multicolumn{2}{p{6cm}|}{3} \\
        \hline
        \multicolumn{2}{|p{6cm}|}{\textbf{Prioridad}} & \multicolumn{2}{p{6cm}|}{P1} \\
        \hline
        \multicolumn{2}{|p{6cm}|}{\textbf{Criterios de aceptación}} & \multicolumn{2}{p{6cm}|}{
            \begin{itemize}
                \item Las clases extra aparecerán, como las clases, en la vista del horario.
                \item Se debe mandar un correo electrónico a los alumnos que afecte.
            \end{itemize}
        } \\
        \hline
    \end{tabular}
    \caption{Historia de usuario HU-16}
    \label{tab:hu_16}
\end{table}

\begin{table}[H]
    \centering
    \begin{tabular}{|p{2cm}|p{4cm}|p{2cm}|p{4cm}|}
        \hline
        \textbf{ID} & HU-17 & \textbf{Nombre} & Crear evento a nivel de facultad \\
        \hline
        \multicolumn{2}{|p{6cm}|}{\textbf{Descripción}} & \multicolumn{2}{p{6cm}|}{Como administrador he de poder crear eventos a nivel de facultad ( charlas, conferencias, exámenes …).} \\
        \hline
        \multicolumn{2}{|p{6cm}|}{\textbf{Estimación}} & \multicolumn{2}{p{6cm}|}{5} \\
        \hline
        \multicolumn{2}{|p{6cm}|}{\textbf{Prioridad}} & \multicolumn{2}{p{6cm}|}{P1} \\
        \hline
        \multicolumn{2}{|p{6cm}|}{\textbf{Criterios de aceptación}} & \multicolumn{2}{p{6cm}|}{
            \begin{itemize}
                \item Se debe especificar la fecha, hora de inicio, hora de fin y tipo de evento.
                \item El evento no debe coincidir con otra clase existente en horario y aula.
                \item El usuario debe ser un administrador.
            \end{itemize}
        } \\
        \hline
    \end{tabular}
    \caption{Historia de usuario HU-17}
    \label{tab:hu_17}
\end{table}

\begin{table}[H]
    \centering
    \begin{tabular}{|p{2cm}|p{4cm}|p{2cm}|p{4cm}|}
        \hline
        \textbf{ID} & HU-18 & \textbf{Nombre} & Registrar días festivos de facultad \\
        \hline
        \multicolumn{2}{|p{6cm}|}{\textbf{Descripción}} & \multicolumn{2}{p{6cm}|}{Como administrador puedo registrar los días festivos de mi facultad.} \\
        \hline
        \multicolumn{2}{|p{6cm}|}{\textbf{Estimación}} & \multicolumn{2}{p{6cm}|}{3} \\
        \hline
        \multicolumn{2}{|p{6cm}|}{\textbf{Prioridad}} & \multicolumn{2}{p{6cm}|}{P1} \\
        \hline
        \multicolumn{2}{|p{6cm}|}{\textbf{Criterios de aceptación}} & \multicolumn{2}{p{6cm}|}{
            \begin{itemize}
                \item El usuario debe ser un administrador.
                \item El día a registrar no puede estar ya registrado.
            \end{itemize}
        } \\
        \hline
    \end{tabular}
    \caption{Historia de usuario HU-18}
    \label{tab:hu_18}
\end{table}

\begin{table}[H]
    \centering
    \begin{tabular}{|p{2cm}|p{4cm}|p{2cm}|p{4cm}|}
        \hline
        \textbf{ID} & HU-19 & \textbf{Nombre} & Recibir alertas de cambios en asignaturas suscritas \\
        \hline
        \multicolumn{2}{|p{6cm}|}{\textbf{Descripción}} & \multicolumn{2}{p{6cm}|}{Como alumno debo poder recibir alertas de cambios en las asignaturas a las que estoy suscrito.} \\
        \hline
        \multicolumn{2}{|p{6cm}|}{\textbf{Estimación}} & \multicolumn{2}{p{6cm}|}{4} \\
        \hline
        \multicolumn{2}{|p{6cm}|}{\textbf{Prioridad}} & \multicolumn{2}{p{6cm}|}{P1} \\
        \hline
        \multicolumn{2}{|p{6cm}|}{\textbf{Criterios de aceptación}} & \multicolumn{2}{p{6cm}|}{
            \begin{itemize}
                \item La alerta debe llegar a los usuarios que estén suscritos a ese grupo de asignatura.
            \end{itemize}
        } \\
        \hline
    \end{tabular}
    \caption{Historia de usuario HU-19}
    \label{tab:hu_19}
\end{table}

\begin{table}[H]
    \centering
    \begin{tabular}{|p{2cm}|p{4cm}|p{2cm}|p{4cm}|}
        \hline
        \textbf{ID} & HU-20 & \textbf{Nombre} & Eliminar evento a nivel de facultad \\
        \hline
        \multicolumn{2}{|p{6cm}|}{\textbf{Descripción}} & \multicolumn{2}{p{6cm}|}{Como profesor / administrador he de poder eliminar los eventos que he creado a nivel de facultad ( charlas, conferencias, exámenes …).} \\
        \hline
        \multicolumn{2}{|p{6cm}|}{\textbf{Estimación}} & \multicolumn{2}{p{6cm}|}{3} \\
        \hline
        \multicolumn{2}{|p{6cm}|}{\textbf{Prioridad}} & \multicolumn{2}{p{6cm}|}{P2} \\
        \hline
        \multicolumn{2}{|p{6cm}|}{\textbf{Criterios de aceptación}} & \multicolumn{2}{p{6cm}|}{
            \begin{itemize}
                \item El usuario debe ser un administrador.
            \end{itemize}
        } \\
        \hline
    \end{tabular}
    \caption{Historia de usuario HU-20}
    \label{tab:hu_20}
\end{table}

\begin{table}[H]
    \centering
    \begin{tabular}{|p{2cm}|p{4cm}|p{2cm}|p{4cm}|}
        \hline
        \textbf{ID} & HU-21 & \textbf{Nombre} & Eliminar día festivo de facultad \\
        \hline
        \multicolumn{2}{|p{6cm}|}{\textbf{Descripción}} & \multicolumn{2}{p{6cm}|}{Como administrador he de poder eliminar los días festivos de mi facultad.} \\
        \hline
        \multicolumn{2}{|p{6cm}|}{\textbf{Estimación}} & \multicolumn{2}{p{6cm}|}{3} \\
        \hline
        \multicolumn{2}{|p{6cm}|}{\textbf{Prioridad}} & \multicolumn{2}{p{6cm}|}{P2} \\
        \hline
        \multicolumn{2}{|p{6cm}|}{\textbf{Criterios de aceptación}} & \multicolumn{2}{p{6cm}|}{
            \begin{itemize}
                \item El usuario debe ser un administrador.
            \end{itemize}
        } \\
        \hline
    \end{tabular}
    \caption{Historia de usuario HU-21}
    \label{tab:hu_21}
\end{table}

\section{Requisitos funcionales}

A partir de las historias de usuario, junto a sus criterios de aceptación, se han extraído los siguientes requisitos funcionales:

\subsection{Gestión de usuarios}

\begin{itemize}
    \item \textbf{RF-1) Gestión de usuarios:} El sistema debe poder registrar usuarios para futuros inicio de sesión y seguimiento de su información de suscripciones a grupos de asignaturas.
    \begin{itemize}
        \item \textbf{RF-1.1) Inicio de sesión:} El sistema debe permitir el inicio de sesión de usuarios mediante correo electrónico institucional y contraseña.
        \item \textbf{RF-2.2) Registro de usuarios:} El sistema debe tener un proceso de registro de usuario.
        \item \textbf{RF-2.3) Completar el registro:} Para completar el registro el sistema debe mandar un mail para confirmar si el usuario se trata de un alumno o de un profesor.
        \item \textbf{RF-2.4) Cambio de nickname:} El sistema debe permitir cambiar el nickname al usuario por uno no usado.
        \item \textbf{RF-2.5) Cambio de contraseña de acceso:} El sistema debe permitir el cambio de la contraseña de acceso.
        \item \textbf{RF-2.6) Dar de baja:} El sistema debe permitir al usuario darse de baja con el objetivo de que no le lleguen más correos relacionados.
        \item \textbf{RF-2.7) Cambio de rol:} EL sistema debe poder facilitar el cambio de rol de profesor a administrador, y viceversa.
    \end{itemize}
\end{itemize}

\subsection{Gestión de horarios académicos}

\begin{itemize}
    \item \textbf{RF-2) Gestión de horarios académicos:} El sistema debe poder obtener la información relacionada con el horario académico de todos los grados de la UGR, para así poder identificar los horarios personalizados de alumnos y docentes a través de un sistema de suscripción a grupos de asignatura.
    \begin{itemize}
        \item \textbf{RF-2.1) Recopilación de horarios:} El sistema debe recopilar la información de horarios académicos de todos los grados de la UGR.
        \item \textbf{RF-2.2) Grados del alumno / profesor:} El sistema debe recoger el grado/ grados académicos que está cursando / impartiendo el alumno / profesor.
        \item \textbf{RF-2.3) Asignaturas del alumno / profesor:} El sistema debe recoger las asignaturas que está cursando / impartiendo el alumno / profesor.
        \item \textbf{RF-2.4) Grupos del alumno / profesor:} El sistema debe recoger los grupos de las asignaturas que está cursando / impartiendo el alumno / profesor.
        \item \textbf{RF-2.5) Eliminar grados del alumno / profesor:} El sistema debe poder eliminar el grado/ grados académicos que está cursando / impartiendo el alumno / profesor.
        \item \textbf{RF-2.6) Eliminar asignaturas del alumno / profesor:} El sistema debe poder eliminar las asignaturas que está cursando / impartiendo el alumno / profesor.
        \item \textbf{RF-2.7) Eliminar grupos del alumno / profesor:} El sistema debe poder eliminar los grupos de las asignaturas que está cursando / impartiendo el alumno / profesor.
        \item \textbf{RF-2.8) Horario personalizado:} El usuario ha de poder acceder a la información de horario académico de los grupos de asignaturas a los que esté suscrito.
        \item \textbf{RF-2.9) Crear clases extra:} El sistema debe permitir al profesor / administrador crear clases extra a las oficiales.
        \item \textbf{RF-2.10) Eliminar clases extra:} El sistema debe permitir al profesor / administrador eliminar clases extra a las oficiales.
        \item \textbf{RF-2.11) Exportar horario a estándar:} El sistema debe poder exportar el horario en formato estándar (.ics).
        \item \textbf{RF-2.12) Sincronizar con Google calendar:} El sistema deberá poder sincronizarse con Google Calendar.
        \item \textbf{RF-2.13) Alertas sobre clases extra:} El sistema debe poder mandar alertas sobre clases extra a los alumnos de ese grupo.
        \item \textbf{RF-2.14) Crear eventos a nivel de facultad:} El sistema debe poder crear eventos a nivel de facultad.
        \item \textbf{RF-2.15) Registrar días festivos de facultad:} El sistema debe poder registrar los días festivos de la facultad.
        \item \textbf{RF-2.16) Alertas de cambios en asignaturas suscritas:} El sistema debe poder mandar alertas de cambios en las asignaturas suscritas.
        \item \textbf{RF-2.17) Eliminar eventos a nivel de facultad:} El sistema debe poder eliminar eventos a nivel de facultad.
        \item \textbf{RF-2.18) Eliminar días festivos de facultad:} El sistema debe poder eliminar días festivos de la facultad.
    \end{itemize}
\end{itemize}

\section{Requisitos no funcionales}

\subsection{Rendimiento}

\begin{itemize}
    \item \textbf{RNF-1.1) Tiempo de respuesta:} El sistema debe responder a las solicitudes de los usuarios en un tiempo máximo de 3 segundos.
    \item \textbf{RNF-1.2) Capacidad de usuarios concurrentes:} El sistema debe soportar un mínimo de 100 usuarios concurrentes sin degradación significativa del rendimiento.
\end{itemize}

\subsection{Usabilidad}

\begin{itemize}
    \item \textbf{RNF-2.1) Interfaz intuitiva:} La interfaz de usuario debe ser fácil de usar y comprender, incluso para usuarios sin experiencia técnica.
    \item \textbf{RNF-2.2) Accesibilidad:} El sistema debe cumplir con las pautas de accesibilidad web (WCAG) para garantizar que sea utilizable por personas con discapacidades.
    \item \textbf{RNF-2.3) Compatibilidad con dispositivos:} El sistema debe ser compatible con cualquier navegador web, y a cualquier resolución.
\end{itemize}

\subsection{Seguridad}

\begin{itemize}
    \item \textbf{RNF-3.1) Autenticación segura:} El sistema debe implementar un mecanismo de autenticación y autorización basado en JWT.
    \item \textbf{RNF-3.2) Protección de datos:} El sistema debe proteger los datos de usuario confidenciales (como contraseñas y correos electrónicos) mediante cifrado y otras medidas de seguridad.
    \item \textbf{RNF-3.3) Autorización:} El sistema debe controlar el acceso a las funciones del sistema según los roles de usuario (administrador, profesor, alumno).
\end{itemize}

\subsection{Mantenibilidad}

\begin{itemize}
    \item \textbf{RNF-4.1) Modularidad:} El sistema debe estar diseñado de forma modular para facilitar el mantenimiento y la actualización.
    \item \textbf{RNF-4.2) Documentación:} El sistema debe estar debidamente documentado para facilitar la comprensión y el mantenimiento del código.
    \item \textbf{RNF-4.3) Pruebas:} El sistema debe incluir pruebas unitarias y de integración para garantizar la calidad del código.
    \item \textbf{RNF-4.4) Descubrimiento:} El sistema ha de tener un servicio de descubrimiento de servicios para facilitar la extensión del sistema.
    \item \textbf{RNF-4.5) Configuración:} El sistema ha de contar con un servidor de configuración para centralizarla.
\end{itemize}

\subsection{Portabilidad}

\begin{itemize}
    \item \textbf{RNF-5.1) Independencia de plataforma:} El sistema debe ser independiente de la plataforma, lo que significa que debe poder ejecutarse en diferentes sistemas operativos y entornos de servidor.
\end{itemize}

\subsection{Disponibilidad}

\begin{itemize}
    \item \textbf{RNF-6.1) Tiempo de actividad:} El sistema debe tener un tiempo de actividad del 99.9\%.
    \item \textbf{RNF-6.2) Recuperación ante fallos:} El sistema debe poder recuperarse de fallos de hardware o software sin pérdida de datos.
\end{itemize}

\section{Requisitos de información}

\section{Validación de los requisitos}

\section{Conclusiones}
En este capítulo concluimos que...