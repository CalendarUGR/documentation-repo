\chapter{Especificación de requisitos}\label{cap:especificación}

\section{Metodología de desarrollo}
En este capítulo explicaremos...

\section{Personas}

\section{Escenarios}

\section{Historias de usuario}

Una historia de usuario es una explicación general e informal de una función de software escrita desde la perspectiva
del usuario final. Su propósito es articular cómo proporcionará una función de software valor al cliente. \cite{atlassian_user_stories}.

Estas no usan un lenguaje técnico y preciso para definir y acotar los requisitos de un sistema, sino que se enfocan en
el usuario final y en cómo este interactuará con el sistema. Por lo tanto, las historias de usuario son una herramienta
de comunicación entre el equipo de desarrollo y el cliente.

En \hyperlink{scrum}{SCRUM} las historias de usuario son una parte fundamental del proceso de desarrollo de software. En
este marco de trabajo, las historias de usuario son utilizadas para definir los requisitos del sistema y son la base para
la planificación y estimación de las tareas a realizar.
\newline\newline
\negritas{¿Por qué son importantes las historias de usuario?}
\begin{itemize}
    \item Centran la atención en el usuario final.
    \item Permiten la colaboración y comunicación entre el equipo de desarrollo y el cliente.
    \item Fomentan soluciones creativas y flexibles.
\end{itemize}
\subsection{Estructura de una historia de usuario}

Las historias de usuario siguen una estructura general simple y clara. 
\newline\newline
\textbf{Como} [tipo de usuario], \textbf{quiero} [realizar una acción], \textbf{para} [obtener un beneficio].

\begin{itemize}
    \item \textbf{Como}: describe el tipo de usuario que está interactuando con el sistema.
    \item \textbf{Quiero}: describe la acción que el usuario desea realizar.
    \item \textbf{Para}: describe el beneficio que el usuario obtendrá al realizar la acción.   
\end{itemize}

Además de esta estructura general, las historias de usuario pueden incluir otros elementos como criterios de aceptación,
prioridad, estimación de esfuerzo, entre otros.

Para se ha definido la siguiente estructura para las historias de usuario:

%% Tabla: Estructura de una historia de usuario
\begin{table}[H]
    \centering
    \begin{tabular}{|p{2cm}|p{4cm}|p{2cm}|p{4cm}|}
        \hline
        \textbf{ID} & Identificador único de la historia de usuario. & \textbf{Nombre} & Nombre de la historia de usuario. \\
        \hline
        \multicolumn{2}{|p{6cm}|}{\textbf{Descripción}} & \multicolumn{2}{p{6cm}|}{Descripción general de la historia de usuario.} \\
        \hline
        \multicolumn{2}{|p{6cm}|}{\textbf{Estimación}} & \multicolumn{2}{p{6cm}|}{Estimación del esfuerzo necesario para completar la historia de usuario. Basado en Planning Poker.} \\
        \hline
        \multicolumn{2}{|p{6cm}|}{\textbf{Prioridad}} & \multicolumn{2}{p{6cm}|}{Acción que el usuario desea realizar. Desde P3 (baja) hasta P0 (alta).} \\
        \hline
        \multicolumn{2}{|p{6cm}|}{\textbf{Criterios de aceptación}} & \multicolumn{2}{p{6cm}|}{Conjunto de condiciones que deben cumplirse para considerar la historia de usuario como completada.} \\
        \hline
    \end{tabular}
    \caption{Estructura de una historia de usuario}
    \label{tab:estructura_historia_usuario}
\end{table}

\begin{comment}
    HU-1) Como usuario he de poder iniciar sesión en el sistema

Criterios de aceptación:

[ ] Para poder iniciar sesión ha de insertar su correo y contraseña.
[ ] Sólo se puede iniciar sesión con correos de la ugr.


HU-2) Como usuario he de poder registrarme en el sistema

Criterios de aceptación:

[ ] El alumno sólo se puede registrar con su correo institucional de la ugr.
[ ] El alumno debe insertar nickname, correo y contraseña.
[ ] La contraseña del alumno ha de ser mayor o igual a 9 caracteres, conteniendo esta una mayúscula y un número como mínimo.
[ ] El registro se ha de completar mediante un link mandado por mail.


HU-3) Como usuario puedo modificar mi nickname.

Criterios de aceptación:

[ ] El alumno no puede cambiar su nickname a otro que exista.
[ ] El alumno no puede modificar su correo electrónico.


HU-4) Como usuario he de poder modificar la contraseña de acceso.

Criterios de aceptación:

[ ] Para poder modificar la contraseña ha de insertar la contraseña anterior.
[ ] Se ha de insertar la nueva contraseña 2 veces, siendo esta  mayor o igual a 9 caracteres, y conteniendo una mayúscula y un número como mínimo.

HU-5) Como usuario he de poder darme de baja del sistema.

Criterios de aceptación:

[ ] Para poder completar la baja ha de escribir su contraseña en un campo de texto.

HU-6) Como administrador he de poder actualizar mi rol a profesor, y viceversa.

Criterios de aceptación:

[ ] Para poder cambiar el rol a profesor he de ser administrador.
[ ] Para poder cambiar el rol a profesor he de ser administrador.

HU-7) Como usuario he de poder seleccionar el grado o grados que estoy cursando. ( realmente los dobles grados oficiales cuentan como uno sólo en BD, pero si estoy cursando, por ejemplo, Matemáticas y Química realmente son dos grados independientes ).

Criterios de aceptación:

[ ] Se pueden seleccionar un máximo de 4 grados.

HU-8) Como usuario he de poder eliminar un grado que ya no esté cursando.

Criterios de aceptación:

[ ] Si el usuario está suscrito a grupos de asignatura de ese grado, se le recordará que también se revocarán sus suscripciones a estos.

HU-9) Como usuario he de poder suscribirme a los grupos de asignaturas a las que quiero hacer seguimiento.

Criterios de aceptación:

[ ] Para hacer seguimiento a un grupo en concreto, el usuario deberá estar cursando el grado al que pertenece.

HU-10) Como usuario he de poder revocar una suscripción a un grupo de asignatura.

Criterios de aceptación:

[ ] El usuario sólo puede revocar suscripciones de grupos a los que está suscrito.

HU-11) Como usuario he de poder ver mi horario de grupos de asignatura en diferentes formatos ( día, semana, mes , año ).

Criterios de aceptación:

[ ] El horario debe mostrar la asignatura, grupo, hora de inicio y fin, profesores del grupo, y aula.
[ ] Las clases sólo deben mostrarse en el rango de fechas en las que se imparten.

HU-12) Como profesor he de poder realizar cambios sobre los grupos de asignatura a los que me he suscrito ( cambio de aula, horario de inicio y fin y profesor ).

Criterios de aceptación:

[ ] El profesor solo podrá realizar cambios en los grupos en los que imparta clase.
[ ] El sistema debe permitir modificar el aula, y el horario.
[ ] Los cambios deben generar alertas para los alumnos suscritos.

HU-13) Como administrador he de poder realizar cambios sobre cualquier grupo de asignatura ( cambio de aula, horario de inicio y fin y profesor ).

Criterios de aceptación:

[ ] El sistema debe permitir modificar el aula, y el horario.
[ ] Los cambios deben generar alertas para los alumnos suscritos.

HU-14) Como profesor / administrador he de poder crear eventos puntuales a nivel de grupo ( clases de recuperación, extra, charlas …).

Criterios de aceptación:

[ ]  Se debe especificar la fecha, hora de inicio, hora de fin y tipo de evento.
[ ] Los eventos deben notificarse automáticamente a los alumnos suscritos.
[ ] Los eventos deben aparecer en el horario del grupo y de los alumnos suscritos.

HU-15) Como usuario he de poder exportar mi horario para usarlo en otros sistemas estándar de calendario.

Criterios de aceptación:

[ ] El usuario podrá exportar su horario en formatos compatibles con sistemas estándar de calendario (.ics).
[ ] La exportación debe incluir todas las asignaturas y eventos del usuario.
[ ] Se debe permitir elegir un rango de fechas para la exportación.
[ ] El archivo generado debe poder descargarse y ser importable en Google Calendar, Outlook, Apple Calendar, etc.

HU-16) Como usuario he de poder sincronizar mi calendario con Google Calendar.

Criterios de aceptación:

[ ] El usuario podrá vincular su cuenta con Google Calendar mediante OAuth.
[ ] Los eventos de su horario deben sincronizarse automáticamente con Google Calendar.
[ ] Se deben reflejar en Google Calendar los cambios realizados en el horario del usuario.
[ ] El usuario debe poder desactivar la sincronización en cualquier momento.

HU-17) Como alumno he de poder ver alertas sobre cambios en las clases de los grupos a los que estoy suscrito ( cambio de aula, horario de inicio y fin y profesor ).

Criterios de aceptación:

[ ] Las alertas aparecerán en la vista principal del horario durante la siguiente semana a partir del cambio.
[ ] Si el cambio es de aula u horario se mandará además un correo electrónico.

HU-18) Como alumno he de poder ver alertas referentes a eventos “esta semana” de mis grupos ( clases de recuperación, extra, charlas …).

Criterios de aceptación:

[ ] Las alertas aparecerán en la vista principal del horario durante la siguiente semana a partir del cambio.
[ ] Los eventos aparecerán, como las clases, en la vista del horario.

\end{comment}

\subsection{Historias de usuario}

\begin{table}[H]
    \centering
    \begin{tabular}{|p{2cm}|p{4cm}|p{2cm}|p{4cm}|}
        \hline
        \textbf{ID} & HU-1 & \textbf{Nombre} & Iniciar sesión \\
        \hline
        \multicolumn{2}{|p{6cm}|}{\textbf{Descripción}} & \multicolumn{2}{p{6cm}|}{Como usuario he de poder iniciar sesión en el sistema.} \\
        \hline
        \multicolumn{2}{|p{6cm}|}{\textbf{Estimación}} & \multicolumn{2}{p{6cm}|}{3} \\
        \hline
        \multicolumn{2}{|p{6cm}|}{\textbf{Prioridad}} & \multicolumn{2}{p{6cm}|}{P0} \\
        \hline
        \multicolumn{2}{|p{6cm}|}{\textbf{Criterios de aceptación}} & \multicolumn{2}{p{6cm}|}{
            \begin{itemize}
                \item Para poder iniciar sesión ha de insertar su correo y contraseña.
                \item Sólo se puede iniciar sesión con correos de la UGR.
            \end{itemize}
        } \\
        \hline
    \end{tabular}
    \caption{Historia de usuario HU-1}
    \label{tab:hu_1}
\end{table}

\begin{table}[H]
    \centering
    \begin{tabular}{|p{2cm}|p{4cm}|p{2cm}|p{4cm}|}
        \hline
        \textbf{ID} & HU-2 & \textbf{Nombre} & Registrarse \\
        \hline
        \multicolumn{2}{|p{6cm}|}{\textbf{Descripción}} & \multicolumn{2}{p{6cm}|}{Como usuario he de poder registrarme en el sistema.} \\
        \hline
        \multicolumn{2}{|p{6cm}|}{\textbf{Estimación}} & \multicolumn{2}{p{6cm}|}{5} \\
        \hline
        \multicolumn{2}{|p{6cm}|}{\textbf{Prioridad}} & \multicolumn{2}{p{6cm}|}{P0} \\
        \hline
        \multicolumn{2}{|p{6cm}|}{\textbf{Criterios de aceptación}} & \multicolumn{2}{p{6cm}|}{
            \begin{itemize}
                \item El alumno sólo se puede registrar con su correo institucional de la UGR.
                \item El alumno debe insertar nickname, correo y contraseña.
                \item La contraseña del alumno ha de ser mayor o igual a 9 caracteres, conteniendo esta una mayúscula y un número como mínimo.
                \item El registro se ha de completar mediante un link mandado por mail.
            \end{itemize}
        } \\
        \hline
    \end{tabular}
    \caption{Historia de usuario HU-2}
    \label{tab:hu_2}
\end{table}

\begin{table}[H]
    \centering
    \begin{tabular}{|p{2cm}|p{4cm}|p{2cm}|p{4cm}|}
        \hline
        \textbf{ID} & HU-3 & \textbf{Nombre} & Modificar nickname \\
        \hline
        \multicolumn{2}{|p{6cm}|}{\textbf{Descripción}} & \multicolumn{2}{p{6cm}|}{Como usuario puedo modificar mi nickname.} \\
        \hline
        \multicolumn{2}{|p{6cm}|}{\textbf{Estimación}} & \multicolumn{2}{p{6cm}|}{2} \\
        \hline
        \multicolumn{2}{|p{6cm}|}{\textbf{Prioridad}} & \multicolumn{2}{p{6cm}|}{P1} \\
        \hline
        \multicolumn{2}{|p{6cm}|}{\textbf{Criterios de aceptación}} & \multicolumn{2}{p{6cm}|}{
            \begin{itemize}
                \item El alumno no puede cambiar su nickname a otro que exista.
                \item El alumno no puede modificar su correo electrónico.
            \end{itemize}
        } \\
        \hline
    \end{tabular}
    \caption{Historia de usuario HU-3}
    \label{tab:hu_3}
\end{table}

\begin{table}[H]
    \centering
    \begin{tabular}{|p{2cm}|p{4cm}|p{2cm}|p{4cm}|}
        \hline
        \textbf{ID} & HU-4 & \textbf{Nombre} & Modificar contraseña \\
        \hline
        \multicolumn{2}{|p{6cm}|}{\textbf{Descripción}} & \multicolumn{2}{p{6cm}|}{Como usuario he de poder modificar la contraseña de acceso.} \\
        \hline
        \multicolumn{2}{|p{6cm}|}{\textbf{Estimación}} & \multicolumn{2}{p{6cm}|}{3} \\
        \hline
        \multicolumn{2}{|p{6cm}|}{\textbf{Prioridad}} & \multicolumn{2}{p{6cm}|}{P1} \\
        \hline
        \multicolumn{2}{|p{6cm}|}{\textbf{Criterios de aceptación}} & \multicolumn{2}{p{6cm}|}{
            \begin{itemize}
                \item Para poder modificar la contraseña ha de insertar la contraseña anterior.
                \item Se ha de insertar la nueva contraseña 2 veces, siendo esta  mayor o igual a 9 caracteres, y conteniendo una mayúscula y un número como mínimo.
            \end{itemize}
        } \\
        \hline
    \end{tabular}
    \caption{Historia de usuario HU-4}
    \label{tab:hu_4}
\end{table}

\begin{table}[H]
    \centering
    \begin{tabular}{|p{2cm}|p{4cm}|p{2cm}|p{4cm}|}
        \hline
        \textbf{ID} & HU-5 & \textbf{Nombre} & Darse de baja \\
        \hline
        \multicolumn{2}{|p{6cm}|}{\textbf{Descripción}} & \multicolumn{2}{p{6cm}|}{Como usuario
        he de poder darme de baja del sistema.} \\
        \hline
        \multicolumn{2}{|p{6cm}|}{\textbf{Estimación}} & \multicolumn{2}{p{6cm}|}{1} \\
        \hline
        \multicolumn{2}{|p{6cm}|}{\textbf{Prioridad}} & \multicolumn{2}{p{6cm}|}{P2} \\
        \hline
        \multicolumn{2}{|p{6cm}|}{\textbf{Criterios de aceptación}} & \multicolumn{2}{p{6cm}|}{
            \begin{itemize}
                \item Para poder completar la baja ha de escribir su contraseña en un campo de texto.
            \end{itemize}
        } \\
        \hline
    \end{tabular}
    \caption{Historia de usuario HU-5}
    \label{tab:hu_5}
\end{table}

\begin{table}[H]
    \centering
    \begin{tabular}{|p{2cm}|p{4cm}|p{2cm}|p{4cm}|}
        \hline
        \textbf{ID} & HU-6 & \textbf{Nombre} & Cambiar rol \\
        \hline
        \multicolumn{2}{|p{6cm}|}{\textbf{Descripción}} & \multicolumn{2}{p{6cm}|}{Como administrador he de poder actualizar mi rol a profesor, y viceversa.} \\
        \hline
        \multicolumn{2}{|p{6cm}|}{\textbf{Estimación}} & \multicolumn{2}{p{6cm}|}{2} \\
        \hline
        \multicolumn{2}{|p{6cm}|}{\textbf{Prioridad}} & \multicolumn{2}{p{6cm}|}{P1} \\
        \hline
        \multicolumn{2}{|p{6cm}|}{\textbf{Criterios de aceptación}} & \multicolumn{2}{p{6cm}|}{
            \begin{itemize}
                \item Para poder cambiar el rol a profesor he de ser administrador.
                \item Para poder cambiar el rol a profesor he de ser administrador.
            \end{itemize}
        } \\
        \hline
    \end{tabular}
    \caption{Historia de usuario HU-6}
    \label{tab:hu_6}
\end{table}

\begin{table}[H]
    \centering
    \begin{tabular}{|p{2cm}|p{4cm}|p{2cm}|p{4cm}|}
        \hline
        \textbf{ID} & HU-7 & \textbf{Nombre} & Seleccionar grados \\
        \hline
        \multicolumn{2}{|p{6cm}|}{\textbf{Descripción}} & \multicolumn{2}{p{6cm}|}{Como usuario he de poder seleccionar el grado o grados que estoy cursando.} \\
        \hline
        \multicolumn{2}{|p{6cm}|}{\textbf{Estimación}} & \multicolumn{2}{p{6cm}|}{3} \\
        \hline
        \multicolumn{2}{|p{6cm}|}{\textbf{Prioridad}} & \multicolumn{2}{p{6cm}|}{P1} \\
        \hline
        \multicolumn{2}{|p{6cm}|}{\textbf{Criterios de aceptación}} & \multicolumn{2}{p{6cm}|}{
            \begin{itemize}
                \item Se pueden seleccionar un máximo de 4 grados.
            \end{itemize}
        } \\
        \hline
    \end{tabular}
    \caption{Historia de usuario HU-7}
    \label{tab:hu_7}
\end{table}

\begin{table}[H]
    \centering
    \begin{tabular}{|p{2cm}|p{4cm}|p{2cm}|p{4cm}|}
        \hline
        \textbf{ID} & HU-8 & \textbf{Nombre} & Eliminar grado \\
        \hline
        \multicolumn{2}{|p{6cm}|}{\textbf{Descripción}} & \multicolumn{2}{p{6cm}|}{Como usuario he de poder eliminar un grado que ya no esté cursando.} \\
        \hline
        \multicolumn{2}{|p{6cm}|}{\textbf{Estimación}} & \multicolumn{2}{p{6cm}|}{2} \\
        \hline
        \multicolumn{2}{|p{6cm}|}{\textbf{Prioridad}} & \multicolumn{2}{p{6cm}|}{P1} \\
        \hline
        \multicolumn{2}{|p{6cm}|}{\textbf{Criterios de aceptación}} & \multicolumn{2}{p{6cm}|}{
            \begin{itemize}
                \item Si el usuario está suscrito a grupos de asignatura de ese grado, se le recordará que también se revocarán sus suscripciones a estos.
            \end{itemize}
        } \\
        \hline
    \end{tabular}
    \caption{Historia de usuario HU-8}
    \label{tab:hu_8}
\end{table}

\begin{table}[H]
    \centering
    \begin{tabular}{|p{2cm}|p{4cm}|p{2cm}|p{4cm}|}
        \hline
        \textbf{ID} & HU-9 & \textbf{Nombre} & Suscribirse a grupos de asignatura \\
        \hline
        \multicolumn{2}{|p{6cm}|}{\textbf{Descripción}} & \multicolumn{2}{p{6cm}|}{Como usuario he de poder suscribirme a los grupos de asignaturas a las que quiero hacer seguimiento.} \\
        \hline
        \multicolumn{2}{|p{6cm}|}{\textbf{Estimación}} & \multicolumn{2}{p{6cm}|}{4} \\
        \hline
        \multicolumn{2}{|p{6cm}|}{\textbf{Prioridad}} & \multicolumn{2}{p{6cm}|}{P1} \\
        \hline
        \multicolumn{2}{|p{6cm}|}{\textbf{Criterios de aceptación}} & \multicolumn{2}{p{6cm}|}{
            \begin{itemize}
                \item Para hacer seguimiento a un grupo en concreto, el usuario deberá estar cursando el grado al que pertenece.
            \end{itemize}
        } \\
        \hline
    \end{tabular}
    \caption{Historia de usuario HU-9}
    \label{tab:hu_9}
\end{table}

\begin{table}[H]
    \centering
    \begin{tabular}{|p{2cm}|p{4cm}|p{2cm}|p{4cm}|}
        \hline
        \textbf{ID} & HU-10 & \textbf{Nombre} & Revocar suscripción a grupo de asignatura \\
        \hline
        \multicolumn{2}{|p{6cm}|}{\textbf{Descripción}} & \multicolumn{2}{p{6cm}|}{Como usuario he de poder revocar una suscripción a un grupo de asignatura.} \\
        \hline
        \multicolumn{2}{|p{6cm}|}{\textbf{Estimación}} & \multicolumn{2}{p{6cm}|}{2} \\
        \hline
        \multicolumn{2}{|p{6cm}|}{\textbf{Prioridad}} & \multicolumn{2}{p{6cm}|}{P1} \\
        \hline
        \multicolumn{2}{|p{6cm}|}{\textbf{Criterios de aceptación}} & \multicolumn{2}{p{6cm}|}{
            \begin{itemize}
                \item El usuario sólo puede revocar suscripciones de grupos a los que está suscrito.
            \end{itemize}
        } \\
        \hline
    \end{tabular}
    \caption{Historia de usuario HU-10}
    \label{tab:hu_10}
\end{table}

\begin{table}[H]
    \centering
    \begin{tabular}{|p{2cm}|p{4cm}|p{2cm}|p{4cm}|}
        \hline
        \textbf{ID} & HU-11 & \textbf{Nombre} & Ver horario de grupos de asignatura \\
        \hline
        \multicolumn{2}{|p{6cm}|}{\textbf{Descripción}} & \multicolumn{2}{p{6cm}|}{Como usuario he de poder ver mi horario de grupos de asignatura en diferentes formatos ( día, semana, mes , año ).} \\
        \hline
        \multicolumn{2}{|p{6cm}|}{\textbf{Estimación}} & \multicolumn{2}{p{6cm}|}{5} \\
        \hline
        \multicolumn{2}{|p{6cm}|}{\textbf{Prioridad}} & \multicolumn{2}{p{6cm}|}{P1} \\
        \hline
        \multicolumn{2}{|p{6cm}|}{\textbf{Criterios de aceptación}} & \multicolumn{2}{p{6cm}|}{
            \begin{itemize}
                \item El horario debe mostrar la asignatura, grupo, hora de inicio y fin, profesores del grupo, y aula.
                \item Las clases sólo deben mostrarse en el rango de fechas en las que se imparten.
            \end{itemize}
        } \\
        \hline
    \end{tabular}
    \caption{Historia de usuario HU-11}
    \label{tab:hu_11}
\end{table}

%% Faltan las historias de usuario de la siuiente iteración

\section{Requisitos funcionales}

\section{Requisitos no funcionales}

\section{Requisitos de información}

\section{Validación de los requisitos}

\section{Conclusiones}
En este capítulo concluimos que...