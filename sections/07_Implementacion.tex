\chapter{Implementación}\label{cap:implementacion}

\section{Iteración 0}

Curso de microsericios en el ecosistema de Spring Boot, reuniones con el director del TFG para definir el sistema.
Reuniones con el secretario de la facultad, con funcionariado de la ETSIIT, y con el director del CEPRUD para conocer  la información con la que cuento.
Pruebas en Spring Boot.

\section{Iteración 1}

Se implementaron todas las historias de usuario relacionadas con la gestón de usuarios y roles, así como la autenticación y autorización. Para ello
se implementarion los servicios user-service, auth-service, mail-service y el api-gateway.
Además se hizo una primera aproximación del scrapping de los datos de horarios académicos de todos los grados de la ugr.

\section{Iteración 2}

Se afina el scrapping, se crea el servicio schedule-consumer-service, y se crea la lógica de suscripciones a grupos, y demás tareas relacionadas. Además se puede extraer el ".ics"
de los horarios personalizados, para ello se implementa el academic-subscription-service.

Se crea además el servidor de descubrimiento de servicios con eureka, y así se investiga como mejorar el rendimiento con balanceo de carga y varias instancias.

\section{Iteración 3}

Fin del backend generando eventos a nivel de grupo y a nivel de facultad. Extracción del ics con clases oficiales, clases extra y eventos de facultad. Sincronización con Google calendar.
Alertas cuando se crean eventos a nicel de grupo (clases extra).
Se crean pruebas unitarias y se realizan pruebas de carga en el servidor de despliegue.
Se dockeriza el sistema y se implementa en un servidor de producción.

\section{Iteración 4}

Comienzo del frontend, se implementan las mismas historias de usuario que en el backend.
Refinamiento del backend, añadidos para complementar el frontend.

\section{Iteración 5}

Se acaba el frontend. Se termina la documentación del proyecto.
Se comienza la presentación del TFG.


