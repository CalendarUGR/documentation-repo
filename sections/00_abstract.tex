\chapter*{Abstract}
This Final Degree Project addresses the design and implementation of a backend system for the centralized management and dynamic distribution of the academic calendar of the University of Granada (UGR). The project stems from the need to modernize access to academic information, overcoming the challenges of data dispersion and the lack of integration with current digital productivity tools.

The proposed solution is based on a containerized microservices architecture, selected for its high scalability, resilience, and ease of maintenance. The core of the system is a subscription platform that allows students and faculty, after secure authentication with their university credentials, to generate personalized calendars. Each user can select the lecture and practical groups corresponding to their enrollment or teaching assignments, obtaining a unified and detailed view that includes subjects, classrooms, and instructors.

A fundamental pillar of the project is interoperability with external calendar services. The system enables users to export and synchronize their personalized schedules with widely used platforms such as Google Calendar. This functionality ensures access to academic information in real-time and from any device, eliminating the need for manual and error-prone lookups.

In conclusion, this project not only develops a functional application but also establishes a technological infrastructure that optimizes time management and significantly enhances the user experience for the entire UGR community, promoting a more efficient, accessible, and centralized academic management.

\vspace{.5cm}

\textbf{Keywords:} Microservices, Containerization, Academic Calendar, Interoperability, Backend, Authentication