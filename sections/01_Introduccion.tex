\chapter{Introducción}\label{cap:introduccion}

En la Universidad de Granada (UGR) no existe en la actualidad un sistema centralizado y personalizado de calendario académico para estudiantes y profesorado. Para confeccionar sus propios itinerarios, los usuarios deben acceder a diversas páginas abiertas en la web institucional, combinando manualmente la información de cada uno de los portales disponibles. Esta situación genera redundancias, posibles errores y un coste agregado en tiempo y esfuerzo.

Con el fin de dar respuesta a esta carencia, este proyecto propone la creación de un servicio de calendario automático y personalizado para la comunidad universitaria, adoptando una arquitectura de microservicios como solución de base. Mediante la descomposición del sistema en servicios independientes —cada uno responsable de una funcionalidad concreta como gestión de usuarios, suscripciones, calendario o autenticación— se facilita la comunicación síncrona y asíncrona entre componentes, la integración de un API Gateway para el enrutamiento y la seguridad, y la aplicación de configuraciones y buenas prácticas que aseguren la disponibilidad y el rendimiento del sistema. Este enfoque modular y distribuido aporta flexibilidad, mantenibilidad y despliegues ágiles, a la vez que confiere resiliencia al aislar posibles fallos en servicios específicos.

En resumen, este Trabajo de Fin de Grado se concibe como una oportunidad para implementar un sistema de calendario académico personalizado basado en microservicios, enfrentándose a retos técnicos reales y contribuyendo directamente a la comunidad universitaria. A lo largo del desarrollo se abordarán aspectos clave como el diseño de APIs claras y bien definidas, la persistencia de datos en múltiples servicios, la consistencia y tolerancia a fallos, y las comunicaciones internas del sistema. Además de consolidar conocimientos teóricos, se evaluará el impacto en los costes de tiempo y gestión para los casi 60 000 alumnos y 4 000 profesores de la UGR, demostrando cómo un servicio bien diseñado puede optimizar procesos administrativos esenciales.

\section{Estructura de la memoria}

La estructura de la memoria se divide en los siguientes capítulos:

\begin{itemize}
    \item \textbf{Capítulo 1: Introducción.} En este capítulo se presenta el contexto general del trabajo, la motivación que ha llevado a su realización, y la estructura que seguirá la memoria.
    
    \item \textbf{Capítulo 2: Descripción del problema.} Este capítulo describe con detalle el problema principal que se pretende resolver, los retos existentes en la gestión de horarios académicos y las limitaciones de los enfoques tradicionales. También se identifican las necesidades de los usuarios y se justifica la necesidad de una solución basada en tecnologías web.
    
    \item \textbf{Capítulo 3: Estado del arte.} Se realiza una revisión exhaustiva de las soluciones existentes en el ámbito de los sistemas de gestión de horarios, tanto comerciales como académicos, así como en el desarrollo de sistemas de información web. Se analizan sus ventajas, limitaciones, y se identifican oportunidades de mejora que el presente trabajo busca aprovechar.
    
    \item \textbf{Capítulo 4: Especificación de requisitos.} Aquí se definen los distintos actores del sistema mediante personas y escenarios de uso. Se detallan los requisitos del sistema agrupados en requisitos funcionales, no funcionales y de información. Además, se presentan historias de usuario que ayudan a comprender cómo interactúan los usuarios con el sistema en distintos contextos.
    
    \item \textbf{Capítulo 5: Planificación, metodología y presupuesto del proyecto.} En este capítulo se describe la planificación temporal del proyecto, especificando las fases y entregas principales. Se justifica la elección de la metodología ágil (por ejemplo, Scrum) y se explica cómo se ha aplicado durante el desarrollo. También se presenta una estimación de costes y recursos necesarios, incluyendo un presupuesto detallado.
    
    \item \textbf{Capítulo 6: Diseño del sistema. Arquitectura, tecnologías y decisiones clave} Se expone el diseño de la arquitectura del sistema basada en microservicios, detallando la organización de los diferentes componentes, el diseño de la API REST, la estructura de las bases de datos utilizadas y el diseño de la interfaz de usuario. Se discuten las decisiones técnicas tomadas y se ilustran mediante diagramas e imágenes.
    
    \item \textbf{Capítulo 7: Implementación.} En este capítulo se documenta el desarrollo del sistema, dividido en los distintos sprints planificados. Se detallan los avances realizados, las herramientas utilizadas y las tecnologías integradas. También se explican los problemas encontrados durante la implementación y las soluciones adoptadas.
    
    \item \textbf{Capítulo 8: Despliegue del sistema.} Aquí se describe el proceso de despliegue del sistema en el servidor proporcionado por la Universidad de Granada (UGR). Se detallan aspectos como la configuración del servidor, la gestión de certificados SSL, la habilitación del protocolo HTTPS, el uso de dominios personalizados, y la realización de pruebas de carga en un entorno realista.
    
    \item \textbf{Capítulo 9: Conclusiones y trabajos futuros.} Se presentan las conclusiones obtenidas a partir del desarrollo del proyecto, destacando los logros alcanzados, las lecciones aprendidas y las dificultades superadas. Asimismo, se proponen líneas de trabajo futuro y posibles mejoras que podrían incorporarse al sistema en versiones posteriores.
    
    \item \textbf{Anexo A: Glosario.} Este anexo recoge una recopilación de términos técnicos, abreviaturas y conceptos relevantes empleados a lo largo del documento, con el objetivo de facilitar la comprensión del lector no especializado.
\end{itemize}