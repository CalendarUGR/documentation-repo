\chapter{Introducción}\label{cap:introduccion}

En la Universidad de Granada (UGR) no existe en la actualidad un sistema centralizado y personalizado de calendario académico para estudiantes y profesorado. Para confeccionar sus propios itinerarios, los usuarios deben acceder a diversas páginas abiertas en la web institucional, combinando manualmente la información de cada uno de los portales disponibles. Esta situación genera redundancias, posibles errores y un coste agregado en tiempo y esfuerzo.

Con el fin de dar respuesta a esta carencia, este proyecto propone la creación de un servicio de calendario automático y personalizado para la comunidad universitaria, adoptando una arquitectura de microservicios como solución de base. Mediante la descomposición del sistema en servicios independientes —cada uno responsable de una funcionalidad concreta como gestión de usuarios, suscripciones, calendario o autenticación— se facilita la comunicación síncrona y asíncrona entre componentes, la integración de un API Gateway para el enrutamiento y la seguridad, y la aplicación de configuraciones y buenas prácticas que aseguren la disponibilidad y el rendimiento del sistema . Este enfoque modular y distribuido aporta flexibilidad, mantenibilidad y despliegues ágiles, a la vez que confiere resiliencia al aislar posibles fallos en servicios específicos.

En resumen, este Trabajo de Fin de Grado se concibe como una oportunidad para implementar un sistema de calendario académico personalizado basado en microservicios, enfrentándose a retos técnicos reales y contribuyendo directamente a la comunidad universitaria. A lo largo del desarrollo se abordarán aspectos clave como el diseño de APIs claras y bien definidas, la persistencia de datos en múltiples servicios, la consistencia y tolerancia a fallos, y la monitorización integral del rendimiento. Además de consolidar conocimientos teóricos, se evaluará el impacto en los costes de tiempo y gestión para los casi 60 000 alumnos y 4 000 profesores de la UGR, demostrando cómo un servicio bien diseñado puede optimizar procesos administrativos esenciales.

\section{Estructura de la memoria}

La estructura de la memoria se divide en los siguientes capítulos:

\begin{itemize}
    \item \textbf{Capítulo 1: Introducción.} En este capítulo se presenta la motivación del trabajo y la estructura de la memoria.
    \item \textbf{Capítulo 2: Descripción del problema.} En este capítulo se describe el problema que se va a resolver en el trabajo.
    \item \textbf{Capítulo 3: Estado del arte.} En este capítulo se presenta un resumen del estado del arte tanto en el ámbito de sistemas de gestión de horarios como en el ámbito de sistemas de información basados en web.
    \item \textbf{Capítulo 4: Especificación de requisitos.} En este capítulo se presentan las personas, escenarios, y los requisitos del sistema en forma de historias de usuario y requisitos funcionales, no funcionales, y de información.
    \item \textbf{Capítulo 5: Planificación.} En este capítulo se presenta la planificación temporal de trabajo, la metodología de desarrollo escogida y su implementación. Además se presenta un presupuesto del trabajo.
    \item \textbf{Capítulo 6: Diseño.} En este capítulo se presenta el diseño de la arquitectura de microservicios, de la API REST Implementada, de las bases de datos y del frontend desarrollado.
    \item \textbf{Capítulo 7: Implementación.} En este capítulo se presenta la implementación del sistema dividido en los sprints realizados.
    \item \textbf{Capítulo 8: Despliegue.} En este capítulo se presenta el despliegue del sistema en el servidor de la UGR con lo que todo ello conlleva ( gestión del servidor, SSL, HTTPS, dominio, pruebas de carga en un entorno real etc.).
    \item \textbf{Capítulo 9: Conclusiones.} En este capítulo se presentan las conclusiones y trabajos futuros planteados.
    \item \textbf{Anexo A: Glosario.} En este anexo se presenta un glosario con las definiciones de términos técnicos utilizados a lo largo del trabajo.
\end{itemize}