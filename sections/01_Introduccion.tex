\chapter{Introducción}\label{cap:introduccion}

\Juanlu[]{Voy a utilizar comentarios de este tipo para la corrección. Si necesitas responder, puedes utilizar tu tag ``Juanmi'' (Ver comentario abajo)}

\Juanmi[]{Ejemplo de comentario tuyo}


\Juanlu[]{La \textbf{intruducción} es una de las secciones más importantes de tu TFG. Es donde vas a captar la atención de los evaluadores, donde se van a hacer una primera composición de lugar y donde comienzan a tener una predisposición positiva o negativa sobre tu trabajo. En su estado actual no conseguimos estos objetivos.}

\Juanlu[]{Lo ideal es que comienzes desde el principio, sin especular, con el problema que intentas abordar (crear un servicio de calendario automático y personalizado). Puedes justificarlo planteando la carencia que tiene en la actualidad la UGR en este sentido y el impacto agregado que esto tiene en la comunidad. En los siguientes párrafos te pego la propuesta de chatgpt a este tipo de estructura en la introducción. Básicamente es para que te sirva como guía de por donde tienen que ir los tiros. Obviamente, el TFG es tu trabajo, puedes discrepar y hacerlo de otra forma.}

\Juanlu[]{\textbf{Restructuración de la introducción (chatgpt):}}

\Juanlu[]{En la Universidad de Granada (UGR) no existe en la actualidad un sistema centralizado y personalizado de calendario académico para estudiantes y profesorado. Para confeccionar sus propios itinerarios, los usuarios deben acceder a diversas páginas abiertas en la web institucional, combinando manualmente la información de cada uno de los portales disponibles. Esta situación genera redundancias, posibles errores y un coste agregado en tiempo y esfuerzo.}

\Juanlu[]{Con el fin de dar respuesta a esta carencia, este proyecto propone la creación de un servicio de calendario automático y personalizado para la comunidad universitaria, adoptando una arquitectura de microservicios como solución de base. Mediante la descomposición del sistema en servicios independientes —cada uno responsable de una funcionalidad concreta como gestión de usuarios, eventos, notificaciones o autenticación— se facilita la comunicación síncrona y asíncrona entre componentes, la integración de un API Gateway para el enrutamiento y la seguridad, y la aplicación de estrategias de balanceo de carga que garanticen la disponibilidad y el rendimiento del sistema. Este enfoque modular y distribuido aporta flexibilidad, mantenibilidad y despliegues ágiles, a la vez que confiere resiliencia al aislar posibles fallos en servicios específicos.}

\Juanlu[]{En resumen, este Trabajo de Fin de Grado se concibe como una oportunidad para implementar un sistema de calendario académico personalizado basado en microservicios, enfrentándose a retos técnicos reales y contribuyendo directamente a la comunidad universitaria. A lo largo del desarrollo se abordarán aspectos clave como el diseño de APIs claras y bien definidas, la persistencia de datos en múltiples servicios, la consistencia y tolerancia a fallos, y la monitorización integral del rendimiento. Además de consolidar conocimientos teóricos, se evaluará el impacto en los costes de tiempo y gestión para los casi 60 000 alumnos y 4 000 profesores de la UGR, demostrando cómo un servicio bien diseñado puede optimizar procesos administrativos esenciales.}

La arquitectura de microservicios ha emergido como un paradigma fundamental en el desarrollo de sistemas complejos y escalables. Su enfoque modular y distribuido ofrece ventajas claras en términos de flexibilidad, mantenibilidad y despliegue independiente. Sin embargo, la implementación exitosa de una arquitectura de microservicios conlleva una serie de desafíos técnicos que resultan atractivos desde una perspectiva de aprendizaje y desarrollo profesional.

Este proyecto de fin de grado se presenta como una oportunidad perfecta para explorar de manera profunda los entresijos de esta arquitectura. Conceptos como la comunicación eficiente entre servicios (síncrona y asíncrona), la implementación de un API Gateway robusto para la gestión centralizada de peticiones, enrutamiento, seguridad ... , la aplicación de mecanismos de seguridad en un entorno distribuido, y las estrategias de balanceo de carga para garantizar la disponibilidad del sistema, representan áreas de gran interés práctico y teórico.

El desarrollo de un sistema de calendario personalizado para la Universidad de Granada, basado en una arquitectura de microservicios, permitirá no solo consolidar conocimientos teóricos, sino también enfrentarse a los retos inherentes a la construcción de un sistema distribuido real. La necesidad de diseñar APIs claras y bien definidas, gestionar la persistencia de datos en múltiples servicios, asegurar la consistencia y la tolerancia a fallos, y monitorizar el rendimiento del sistema en su conjunto, proporcionarán una experiencia de aprendizaje invaluable. En definitiva, este proyecto se concibe como un vehículo para comprender y aplicar de manera práctica los principios y las mejores prácticas asociadas a la arquitectura de microservicios, un campo con una alta demanda en la industria del desarrollo de software.

\section{Estructura de la memoria}

La estructura de la memoria se divide en los siguientes capítulos:

\begin{itemize}
    \item \textbf{Capítulo 1: Introducción.} En este capítulo se presenta la motivación del trabajo y la estructura de la memoria.
    \item \textbf{Capítulo 2: Descripción del problema.} En este capítulo se describe el problema que se va a resolver en el trabajo.
    \item \textbf{Capítulo 3: Estado del arte.} En este capítulo se presenta un resumen del estado del arte tanto en el ámbito de sistemas de gestión de horarios como en el ámbito de sistemas de información basados en web.
    \item \textbf{Capítulo 4: Especificación de requisitos.} En este capítulo se presentan las personas, escenarios, y los requisitos del sistema en forma de historias de usuario y requisitos funcionales, no funcionales, y de información.
    \item \textbf{Capítulo 5: Planificación.} En este capítulo se presenta la planificación temporal de trabajo, la metodología de desarrollo escogida y su implementación. Además se presenta un presupuesto del trabajo.
    \item \textbf{Capítulo 6: Diseño.} En este capítulo se presenta el diseño de la arquitectura de microservicios, de la API REST Implementada, de las bases de datos y del frontend desarrollado.
    \item \textbf{Capítulo 7: Implementación.} En este capítulo se presenta la implementación del sistema dividido en los sprints realizados.
    \item \textbf{Capítulo 8: Despliegue.} En este capítulo se presenta el despliegue del sistema en el servidor de la UGR con lo que todo ello conlleva ( gestión del servidor, SSL, HTTPS, dominio, etc.).
    \item \textbf{Capítulo 9: Conclusiones.} En este capítulo se presentan las conclusiones y trabajos futuros planteados.
    \item \textbf{Anexo A: Glosario.} En este anexo se presenta un glosario con las definiciones de términos técnicos utilizados a lo largo del trabajo.
\end{itemize}