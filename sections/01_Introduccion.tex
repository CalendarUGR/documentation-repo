\chapter{Introducción}\label{cap:introduccion}

\section{Motivación}

La arquitectura de microservicios ha emergido como un paradigma fundamental en el desarrollo de sistemas complejos y escalables en la actualidad. Su enfoque modular y distribuido ofrece ventajas significativas en términos de flexibilidad, mantenibilidad y despliegue independiente. Sin embargo, la implementación exitosa de una arquitectura de microservicios conlleva una serie de desafíos técnicos que resultan particularmente atractivos desde una perspectiva de aprendizaje y desarrollo profesional.

Este proyecto de fin de grado se presenta como una oportunidad idónea para explorar en profundidad los entresijos de esta arquitectura. Conceptos como la comunicación eficiente entre servicios (síncrona y asíncrona), la implementación de un API Gateway robusto para la gestión centralizada de peticiones, la aplicación de mecanismos de seguridad coherentes en un entorno distribuido, y las estrategias de balanceo de carga para garantizar la disponibilidad y el rendimiento del sistema, representan áreas de gran interés práctico y teórico.

El desarrollo de un sistema de calendario personalizado para la Universidad de Granada, basado en una arquitectura de microservicios, permitirá no solo consolidar conocimientos teóricos, sino también enfrentarse a los retos inherentes a la construcción de un sistema distribuido real. La necesidad de diseñar APIs claras y bien definidas, gestionar la persistencia de datos en múltiples servicios, asegurar la consistencia y la tolerancia a fallos, y monitorizar el rendimiento del sistema en su conjunto, proporcionarán una experiencia de aprendizaje invaluable. En definitiva, este proyecto se concibe como un vehículo para comprender y aplicar de manera práctica los principios y las mejores prácticas asociadas a la arquitectura de microservicios, un campo con una alta demanda en la industria del desarrollo de software.

\section{Estructura de la memoria}

La estructura de la memoria se divide en los siguientes capítulos:

\begin{itemize}
    \item \textbf{Capítulo 1: Introducción.} En este capítulo se presenta la motivación del trabajo y la estructura de la memoria.
    \item \textbf{Capítulo 2: Descripción del problema.} En este capítulo se describe el problema que se va a resolver en el trabajo.
    \item \textbf{Capítulo 3: Estado del arte.} En este capítulo se presenta un resumen del estado del arte tanto en el ámbito de sistemas de gestión de horarios como en el ámbito de sistemas de información basados en web.
    \item \textbf{Capítulo 4: Especificación de requisitos.} En este capítulo se presentan las personas, escenarios, y los requisitos del sistema en forma de historias de usuario y requisitos funcionales, no funcionales, y de información.
    \item \textbf{Capítulo 5: Planificación.} En este capítulo se presenta la planificación temporal de trabajo, la metodología de desarrollo escogida y su implementación. Además se presenta un presupuesto del trabajo.
    \item \textbf{Capítulo 6: Diseño.} En este capítulo se presenta el diseño de la arquitectura de microservicios, de la API REST Implementada, de las bases de datos y del frontend desarrollado.
    \item \textbf{Capítulo 7: Implementación.} En este capítulo se presenta la implementación del sistema dividido en los sprints realizados.
    \item \textbf{Capítulo 8: Despliegue.} En este capítulo se presenta el despliegue del sistema en el servidor de la UGR con lo que todo ello conlleva ( gestión del servidor, SSL, HTTPS, dominio, etc.).
    \item \textbf{Capítulo 9: Conclusiones.} En este capítulo se presentan las conclusiones y trabajos futuros planteados.
    \item \textbf{Anexo A: Glosario.} En este anexo se presenta un glosario con las definiciones de términos técnicos utilizados a lo largo del trabajo.
\end{itemize}