\chapter{Despliegue del sistema}\label{cap:despliegue}

\section{Contenerización del sistema}

Pasos para contenerizar el sistema:

\begin{enumerate}
    \item Crear una red en docker:
    \begin{lstlisting}[language=bash]
        docker network create calendarugr
    \end{lstlisting}
    \item Generar los .jar de los microservicios, sin pasar los tests para una construcción sin conflictos para los servicios que ya están contenerizados:
    \begin{lstlisting}[language=bash]
        ./mvnw clean package -DskipTests
    \end{lstlisting}
    \item Crear las imágenes de los microservicios (Ej imágen de Eureka service):
    \begin{lstlisting}[language=bash]
        FROM amazoncorretto:21-alpine-jdk
        WORKDIR /app
        EXPOSE 8761
        COPY ./target/eureka-service-0.0.1-SNAPSHOT.jar eureka-service.jar

        ENTRYPOINT ["java", "-jar", "eureka-service.jar"]
    \end{lstlisting}
    \item Construir la imagen de docker:
    \begin{lstlisting}[language=bash]
        docker build -t eureka-service .
    \end{lstlisting}
    \item Para levantar los contenedores uno a uno (Ej levantando el contenedor de Eureka):
    \begin{lstlisting}[language=bash]
        docker run -d --name eureka-service --network calendarugr -p 8761:8761 eureka-service
    \end{lstlisting}
    Si fuera necesario añadir variables de entorno, se puede hacer con el flag \texttt{-e}:
    \begin{lstlisting}[language=bash]
        docker run -d --name eureka-service --network calendarugr -p 8761:8761 \
            -e EUREKA_SERVER_URL=http://eureka-service:8761/eureka/ \
            eureka-service
    \end{lstlisting}
    \item Bajar las imágenes oficiales de mysql:8.0.41 y mongo:6.0.4:
    \begin{lstlisting}[language=bash]
        docker pull mysql:8.0.41
        docker pull mongo:latest
    \end{lstlisting}
    \item Para levantar contenedores con variables de entorno (Ej levantando el contenedor de Mysql):
    \item \begin{lstlisting}[language=bash]
        docker run -p 3307:3306 --network calendarugr \
            -e MYSQL_ROOT_PASSWORD=...\
            -e MYSQL_USER=... \
            -e MYSQL_PASSWORD=... \
            -v /home/juanmi/mysql-scripts/init.sql:/docker-entrypoint-initdb.d/init.sql \
            --name mysql \
            mysql:8.0.41
    \end{lstlisting}
    \item El init.sql es un script que se ejecuta al iniciar el contenedor de Mysql, y se utiliza para crear la base de datos y las tablas necesarias para el funcionamiento del sistema. El script se encuentra en la carpeta \texttt{mysql-scripts} del proyecto.
    \begin{lstlisting}[language=sql]
        CREATE DATABASE IF NOT EXISTS DB_USER_SERVICE;
        CREATE DATABASE IF NOT EXISTS DB_SCHEDULE_CONSUMER_SERVICE;

        GRANT ALL PRIVILEGES ON DB_USER_SERVICE.* TO 'calendarugr'@'%';
        GRANT ALL PRIVILEGES ON DB_SCHEDULE_CONSUMER_SERVICE.* TO 'calendarugr'@'%';
        FLUSH PRIVILEGES;
    \end{lstlisting}
    \item Para levantar el contenedor de Mongo:
    \begin{lstlisting}[language=bash]
        docker run -d --name mongodb \
            -p 27018:27017 \
            --network calendarugr \
            -e MONGO_INITDB_ROOT_USERNAME=admin \
            -e MONGO_INITDB_ROOT_PASSWORD=CalendarUGR@2025 \
            mongo:6.0.4
    \end{lstlisting}
    \item Funciona, para mayor comodidad creamos un docker compose para levantar todos los servicios a la vez:
    

\end{enumerate}
