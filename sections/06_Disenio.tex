\chapter{Diseño del sistema y tecnologías escogidas}\label{cap:disenio}

\section{Arquitectura del sistema}

\subsection{Arquitectura de microservicios}

\subsection{Tecnologías y Frameworks}

\subsection{Diseño de la base de datos}

\subsection{Diseño de la API}

\section{Diseño de la Interfaz de Usuario (UI) y la Experiencia del Usuario (UX)}

En cuanto a la parte visual del proyecto, se tuvo en mente desde el principio tener una interfaz usable, intuitiva y accesible, de manera que se facilitara lo máximo posible el acceso a la información del horario personalizado.
\newline\newline
Para ello, se optó por un diseño minimalista, con una paleta de colores clara y un uso moderado de imágenes. Además se utilizó la tipografía ``Segoe UI'' por su diseño moderno con letras redondeadas y diseño limpio que se ve bien en pantallas y papel.
\newline
\begin{center}
\begin{minipage}{0.6\textwidth}
    \begin{itemize}
        \item Color primario: \#b82d2a \colorbox[HTML]{b82d2a}{\hspace{1.5em}} \vspace{0.7em}
        \item Color secundario: \#e4afae \colorbox[HTML]{e4afae}{\hspace{1.5em}} \vspace{0.7em}
        \item Color de fondo: \#f5f5f5 \colorbox[HTML]{f5f5f5}{\hspace{1.5em}} \vspace{0.7em}
        \item Color de texto: \#333333 \colorbox[HTML]{333333}{\hspace{1.5em}}
    \end{itemize}
\end{minipage}%
\hfill
\begin{minipage}{0.35\textwidth}
    \centering
    \captionsetup{justification=centering}
    \includegraphics[width=0.8\textwidth]{figures/logo.png}
    \captionof{figure}{Logo de TempusUGR}
    \label{fig:logo}
\end{minipage}
\end{center}
