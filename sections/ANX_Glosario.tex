\chapter*{Anexo: Glosario}
\addcontentsline{toc}{chapter}{Anexo: Glosario}

A continuación se presenta un glosario con las definiciones de términos técnicos utilizados a lo largo del trabajo:

\begin{description}
    \item [\hypertarget{scrum}{SCRUM}]: es un marco de trabajo ágil para el desarrollo de software. Se basa en la iteración y la colaboración entre los miembros del equipo de desarrollo.
    \item [\hypertarget{backlog}{Backlog}]: es una lista priorizada de tareas y requisitos que deben completarse en un proyecto. El backlog se utiliza para planificar el trabajo en cada sprint.
    \item [\hypertarget{lms}{LMS}]: es un sistema de gestión de aprendizaje. Se utiliza para administrar, documentar, rastrear, informar y entregar cursos de formación. Un ejemplo de LMS es Moodle, que es un sistema de gestión de aprendizaje de código abierto.
    \item [\hypertarget{docker}{Docker}]: es una plataforma de software que permite crear, desplegar y ejecutar aplicaciones en contenedores. Los contenedores son entornos ligeros y portátiles que permiten ejecutar aplicaciones de manera aislada del sistema operativo subyacente.
    \item [\hypertarget{microservicios}{Microservicios}]: es un estilo arquitectónico que estructura una aplicación como un conjunto de servicios pequeños y autónomos. Cada servicio se ejecuta en su propio proceso y se comunica con otros servicios a través de APIs.
    \item [\hypertarget{api}{API}]: es un conjunto de definiciones y protocolos que permiten la comunicación entre diferentes sistemas. Las APIs permiten que diferentes aplicaciones se comuniquen entre sí y compartan datos.
    \item [\hypertarget{backend}{Backend}]: es la parte de una aplicación que se encarga de la lógica de negocio y el acceso a los datos. El backend se ejecuta en un servidor y se comunica con el frontend a través de APIs.
    \item [\hypertarget{frontend}{Frontend}]: es la parte de una aplicación que se encarga de la interfaz de usuario y la interacción con el usuario. El frontend se ejecuta en el navegador del usuario y se comunica con el backend a través de APIs.
    \item [\hypertarget{ssl}{SSL}]: es un protocolo de seguridad que se utiliza para establecer una conexión segura entre un servidor y un cliente. SSL cifra los datos que se envían entre el servidor y el cliente, lo que protege la información sensible de ser interceptada por terceros.
    \item [\hypertarget{https}{HTTPS}]: es una versión segura de HTTP. HTTPS utiliza SSL para cifrar los datos que se envían entre el servidor y el cliente, lo que protege la información sensible de ser interceptada por terceros.
    \item [\hypertarget{ui}{UI}]: es la interfaz de usuario. Se refiere a la parte de una aplicación con la que el usuario interactúa. La UI incluye elementos como botones, menús y formularios.
    \item [\hypertarget{ux}{UX}]: es la experiencia del usuario. Se refiere a la forma en que un usuario interactúa con una aplicación y cómo se siente al hacerlo. La UX incluye aspectos como la usabilidad, la accesibilidad y la satisfacción del usuario.
    \item [\hypertarget{jwt}{JWT}]: es un estándar abierto que define un formato compacto y autónomo para transmitir información de forma segura entre partes como un objeto JSON. Esta información puede ser verificada y confiable porque está firmada digitalmente.
    \item [\hypertarget{sso}{SSO}]: es un proceso de autenticación que permite a un usuario acceder a múltiples aplicaciones con una sola sesión de inicio de sesión. SSO simplifica la gestión de credenciales y mejora la experiencia del usuario al reducir la necesidad de recordar múltiples contraseñas.
    \item [\hypertarget{rest}{REST}]: es un estilo arquitectónico para diseñar servicios web. REST se basa en el uso de HTTP y utiliza los métodos HTTP (GET, POST, PUT, DELETE) para realizar operaciones sobre recursos.
    \item [\hypertarget{graphql}{GraphQL}]: es un lenguaje de consulta para APIs y un entorno de ejecución para ejecutar esas consultas con los datos existentes. GraphQL permite a los clientes solicitar solo los datos que necesitan, lo que reduce la cantidad de datos transferidos entre el cliente y el servidor.
    \item [\hypertarget{grpc}{gRPC}]: es un marco de trabajo de código abierto que permite la comunicación entre aplicaciones distribuidas. gRPC utiliza HTTP/2 para la comunicación y Protocol Buffers como formato de serialización de datos.
    \item [\hypertarget{reverseproxy}{Reverse Proxy}]: es un servidor que actúa como intermediario entre los clientes y uno o más servidores de backend. El reverse proxy recibe las solicitudes de los clientes y las reenvía a los servidores de backend, lo que permite distribuir la carga y mejorar la seguridad.
    \item [\hypertarget{cronjob}{Cron Job}]: es una tarea programada que se ejecuta automáticamente en un servidor en intervalos regulares. Los cron jobs se utilizan para realizar tareas de mantenimiento, como copias de seguridad, limpieza de registros y actualizaciones de datos.
\end{description}

\endinput