\chapter*{Anexo A: Glosario}
\addcontentsline{toc}{chapter}{Anexo A: Glosario}

A continuación se presenta un glosario con las definiciones de términos técnicos utilizados a lo largo del trabajo:

\begin{description}
    \item[\hypertarget{calendarUGR}{CalendarUGR}] Sistema de gestión personalizada de horarios académicos desarrollado para la Universidad de Granada.
    \item[\hypertarget{api}{API}] Interfaz de Programación de Aplicaciones (Application Programming Interface). Conjunto de reglas y especificaciones que permiten que las aplicaciones se comuniquen entre sí.
    \item[\hypertarget{frontend}{Frontend}] Parte del desarrollo de un software con la que interactúa el usuario, como la interfaz gráfica.
    \item[\hypertarget{backend}{Backend}] Parte del desarrollo de un software que gestiona la lógica de negocio, las bases de datos, y la comunicación entre servidores.
    \item[\hypertarget{uiux}{UI/UX}] Interfaz de Usuario (UI) y Experiencia de Usuario (UX). Conceptos relacionados con el diseño visual y la interacción del usuario con el sistema.
    \item[\hypertarget{orm}{ORM}] Mapeo Relacional de Objetos (Object-Relational Mapping). Técnica para convertir datos entre sistemas incompatibles utilizando orientaciones a objetos.
    \item[\hypertarget{basededatos}{Base de Datos Relacional}] Tipo de base de datos que organiza los datos en tablas interrelacionadas y utiliza el lenguaje SQL para la consulta de los mismos.
    \item[\hypertarget{json}{JSON}] Notación de Objetos de JavaScript (JavaScript Object Notation). Formato de intercambio de datos ligero y fácil de leer.
    \item[\hypertarget{scraping}{Scraping}] Técnica de extracción de datos de sitios web, generalmente automatizada, para recopilar información específica.
    \item[\hypertarget{scheduling}{Algoritmo de Scheduling}] Algoritmo utilizado para asignar eventos o actividades a un calendario de manera óptima, teniendo en cuenta restricciones y preferencias del usuario.
    \item[\hypertarget{responsividad}{Responsividad}] Característica de una aplicación que se adapta de manera eficiente a diferentes tamaños y resoluciones de pantalla.
    \item[\hypertarget{cache}{Cache}] Memoria de acceso rápido utilizada para almacenar datos frecuentemente utilizados, mejorando el rendimiento de la aplicación.
    \item[\hypertarget{oauth}{OAuth}] Protocolo de autorización que permite a las aplicaciones acceder a recursos en nombre de un usuario sin compartir las credenciales.
    \item[\hypertarget{git}{Git}] Sistema de control de versiones distribuido utilizado para gestionar el desarrollo de proyectos de software.
\end{description}

\endinput