\chapter{Conclusiones y trabajos futuros}\label{cap:conclusiones}

\Juanmi[]{Punto a revisar}

El sistema completo, ``TempusUGR'', ha sido desarrollado y desplegado con éxito. Además de estar preparado para un uso normal, se han implementado varias características que lo hacen un sistema robusto y escalable. 

\section{Objetivos cumplidos}
Todos los objetivos planteados al inicio del proyecto se han cumplido, aunque algunos de ellos han evolucionado a lo largo del desarrollo. En particular, se ha logrado:
\begin{itemize}
    \item Desarrollar un sistema de gestión de horarios para la Universidad de Granada que permita a los estudiantes y profesores visualizar sus horarios de manera eficiente.
    \item Proveer al usuario un archivo ``ical'' que pueda ser importado en aplicaciones de calendario como Google Calendar, Apple Calendar, Outlook, etc.
    \item Permitir la sincronización de los horarios con aplicaciones de calendario, facilitando la gestión del tiempo y la planificación de actividades académicas.
    \item Agilizar el proceso de comunicaciones de eventos tanto de grupo como de facultades, a nivel de la UGR.
    \item Implementar una interfaz de usuario intuitiva y fácil de usar, que permita a los usuarios interactuar con el sistema de manera eficiente.
    \item Aprender sobre la arquitectura de microservicios, y sobre varias tecnologías relativas a este tipo de arquitectura.
    \item Desplegar el sistema en un entorno de producción, utilizando herramientas de contenedorización y orquestación como Docker y Docker Compose.
\end{itemize}

\section{Dificultades y resolución}

El proyecto ha supuesto un reto significativo tanto a nivel técnico como a nivel de gestión. Algunas de las dificultades encontradas incluyen:
\begin{itemize}
    \item El sistema se planteó inicialmente, además de como una plataforma de visualización de horario académico y creador de eventos a nivel de grupo y facultad, como un apoyo a las sedes de la UGR para la subida de información de horarios a una única plataforma de manera estándar, esto es lo que se plantea en la descripción incial del proyecto \textit{La Secretaría General dispondrá de herramientas avanzadas de parsing para cargar y actualizar los horarios a partir de los archivos disponibles, asegurando que los calendarios estén siempre actualizados.}. Tras la primera entrevista con Jesús García Miranda, Secretario de la ETSIIT y encargado de la gestión de horarios, se decidió cambiar de dirección hacia un sistema que utilizara los datos ya publicados en un punto central (en este caso Grados UGR).
    Esta decisión se tomó por la falta de información en cuanto a la generación, formato y fechas relativas a la subida de horarios, así como por la falta de un sistema centralizado que permitiera a las sedes subir sus horarios de manera estandarizada.
    Sin embargo, esta decisión también ha permitido centrar el proyecto en una solución más escalable y flexible, que puede adaptarse a futuras necesidades de la UGR. Para la actualización de los datos en base de datos se implementa el ``Cron Job'' que se encarga de renovar la base de datos de forma periódica.

    \item Desde un principio se plantearon varias opciones para la implementación del sistema. Desde solicitar una vista o API a la base de datos de los horarios y calendario académico, desde integrar el sistema de autenticación institucional u obtener la información relativa a las matriculaciones. Al no poder tener acceso a ninguna de estas opciones, se optó por soluciones alternativas e ``independientes'' de la UGR, como la utilización de los datos publicados en Grados UGR y la implementación de un sistema de autenticación propio.
    \item Aunque se realizaron varias reuniones con el director del TFG, Juan Luis Jiménez Laredo, para concretar el alcance y requisitos del proyecto, es inevitable que con la evolución de este se hayan presentado nuevos casos de uso y funcionalidades que no estaban contemplados inicialmente. Esto ha requerido una adaptación constante del proyecto, que gracias a la metodología ágil utilizada, se ha podido gestionar de manera eficiente.
\end{itemize}

Todo el desarrollo del sistema (repositorios del backend, frontend, despliegue y documentación) se puede consultar de manera centralizada en la organización de GitHub [\href{https://github.com/TempusUGR}{TempusUGR}]. De este modo se facilita la colaboración y el mantenimiento del proyecto, permitiendo que otros desarrolladores puedan contribuir y mejorar el sistema en el futuro.
Además se puede visualizar también a través de su propio ``Github Project'' [\href{https://github.com/orgs/TempusUGR/projects/3}{TempusUGR Project}] tanto los tableros de tareas, issues y pull requests del proyecto, como el ``roadmap'' con la temporización real de las tareas y funcionalidades implementadas.

\section{Mejoras posibles y trabajos futuros}

Aunque el sistema ha sido desarrollado y desplegado con éxito, siempre hay margen para mejoras y nuevas funcionalidades. Algunas de las posibles mejoras y trabajos futuros incluyen:
\begin{itemize}
    \item Integración con otros sistemas institucionales de la UGR, como la API de los horarios, autenticación institucional, matriculaciones, etc. Esto permitiría una mayor interoperabilidad y una mejor experiencia de usuario.
    Además la posibilidad de integrar TempusUGR con plataformas establecidas como Prado UGR o SWAD , podría facilitar la adopción del sistema por parte de los usuarios y mejorar la experiencia general.
    \item Implementación de un sistema de notificaciones para alertar a los usuarios sobre cambios en sus horarios de clases oficiales.
    \item Optimización del rendimiento del sistema para manejar un mayor volumen de usuarios y datos, asegurando que el sistema sea escalable y eficiente.
    \item Implementación  de monitorización y estadísticas para analizar el uso del sistema y detectar posibles problemas o áreas de mejora.
    \item Integrar sistemas de Integración Continua/Despliegue Continuo (CI/CD) para facilitar el desarrollo y despliegue de nuevas funcionalidades, asegurando que el sistema se mantenga actualizado y libre de errores. Esto permitiría también una mayor colaboración entre los desarrolladores y una mejor gestión del código fuente.
\end{itemize}
