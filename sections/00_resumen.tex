\chapter*{Resumen}
\Juanmi[]{Revisar resumen y palabras clave}
El presente Trabajo Fin de Grado aborda el diseño e implementación de un sistema backend para la gestión centralizada y la distribución dinámica del calendario académico de la Universidad de Granada. El proyecto nace de la necesidad de modernizar el acceso a la información académica, superando los desafíos de la dispersión de datos y la falta de integración con las herramientas de productividad digital actuales.

La solución se fundamenta en una arquitectura de microservicios contenerizados, seleccionada por su alta escalabilidad, resiliencia y facilidad de mantenimiento. El núcleo del sistema es una plataforma de suscripción que permite a estudiantes y profesores, previa autenticación segura con sus credenciales universitarias, generar calendarios personalizados. Cada usuario puede seleccionar los grupos de teoría y prácticas correspondientes a su matrícula o docencia, obteniendo una vista unificada y detallada que incluye asignaturas, aulas y profesorado.

Un pilar fundamental del proyecto es la interoperabilidad con servicios de calendario externos. El sistema permite a los usuarios exportar y sincronizar sus horarios personalizados con plataformas de amplio uso como Google Calendar. Esta funcionalidad garantiza el acceso a la información académica en tiempo real y desde cualquier dispositivo, eliminando la necesidad de consultas manuales y propensas a errores.

En definitiva, este trabajo no solo desarrolla una aplicación funcional, sino que establece una infraestructura tecnológica que optimiza la gestión del tiempo y mejora significativamente la experiencia de usuario para toda la comunidad de la UGR, promoviendo una gestión académica más eficiente, accesible y centralizada.

\vspace{.5cm}

\textbf{Palabras clave:} Microservicios, Contenerización, Calendario Académico, Interoperabilidad, Backend, Autenticación 