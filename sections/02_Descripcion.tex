\chapter{Descripción del problema}\label{cap:descripcion}

La Universidad de Granada (UGR), con su amplia oferta formativa y su elevado número de usuarios —alrededor de 60 000 estudiantes y 4 000 profesores \cite{webPromocionUGR}— depende de una gestión de horarios académicos ágil y precisa para garantizar la correcta organización de sus actividades docentes. Sin embargo, en la práctica este proceso adolece de dispersión y falta de personalización: cada usuario debe rastrear múltiples portales en la web institucional para construir su propio itinerario, lo que no solo incrementa la probabilidad de errores y solapamientos, sino que también genera un coste oculto en tiempo y recursos cuya magnitud analizaremos en detalle en secciones posteriores.

\section{Contexto y problemática}

La Universidad de Granada (UGR) es una de las universidades más grandes de España, con una amplia oferta académica y un gran número de estudiantes y profesores. La gestión de horarios académicos es un aspecto crítico para el funcionamiento eficiente de la universidad. Sin embargo, la UGR podría mejorar en este campo en varios aspectos:

\begin{itemize}
    \item \textbf{Falta de personalización:} Actualmente, los estudiantes y profesores tienen acceso a un horario general que no se adapta a sus necesidades específicas. Esto dificulta la planificación y organización de su tiempo.
    \item \textbf{Dificultad en la gestión de cambios:} Los cambios en los horarios y eventos académicos (tutorías, clases de recuperación, charlas, etc.) no se comunican de manera efectiva a los estudiantes y profesores, lo que puede llevar a confusiones y malentendidos.
    \item \textbf{Integración con servicios externos:} La falta de integración con servicios de calendario externos como Google Calendar limita la accesibilidad y la organización del horario académico.
\end{itemize}

\section{Estimación del coste agregado}

\Juanlu[]{Sería interesante introducir aquí una sección de \textbf{Estimación del coste agregado}, similar a lo que te he esbozado en chatgpt (compartí el enlace contigo) y teniendo en cuenta las estadísticas numéricas recogidas con las respuestas de los miembros del departamento de Ingeniería de Computadores, Automática y Robótica.}

\Juanlu[]{A continuación una estimación rápida de costes con chatgpt: https://chatgpt.com/share/682f2acf-7a38-8005-ad48-20eca0d1565c. Sin prisa, échale un vistazo a esto que creo que puede venir bien para la documentación: https://docs.google.com/document/d/1oqd1PTm7L2w4XB84V2NTlQKTswEQ0i68qcD0HzrXbKo/edit?usp=sharing}

\section{Solución propuesta}

Partiendo de la problemática observada, se propone la elaboración de un sistema de gestión personalizada de horarios académicos para los grados de la Universidad de Granada. 
Esto permitirá a los estudiantes y profesores acceder a su información horaria de manera centralizada y con comunicaciones efectivas sobre cambios y eventos académicos. Además, la integración con servicios de calendario externos como Google Calendar mejorará la accesibilidad y la organización del horario académico.

\section{Restricciones}

A fin de implementar la solución propuesta de manera eficaz y garantizar que responda a las necesidades detectadas en el contexto de la UGR, es imprescindible que el sistema cumpla una serie de restricciones técnicas y funcionales. Estas restricciones aseguran que la solución sea completa, segura, accesible, compatible, escalable e integrable con otros servicios, permitiendo así su adopción y correcto funcionamiento en el entorno universitario.

\begin{itemize}
    \item \textbf{Completitud:} El sistema debe ser capaz de gestionar todos los grados y asignaturas de la UGR.
    \item \textbf{Seguridad:} El sistema debe manejar la mínima información privada posible para un funcionamiento normal.
    \item \textbf{Accesibilidad:} El sistema debe ser accesible desde cualquier dispositivo con conexión a Internet.
    \item \textbf{Compatibilidad:} El sistema debe ser compatible con los navegadores web más utilizados.
    \item \textbf{Escalabilidad:} El sistema debe ser capaz de manejar un gran número de usuarios y solicitudes simultáneas.
    \item \textbf{Integración:} El sistema debe ser capaz de integrarse con servicios de calendario externos como Google Calendar.
\end{itemize}

\section{Objetivos del proyecto}

En esta sección se detallan los objetivos que guiarán el desarrollo del sistema.
Se dividen en un objetivo principal, que establece la meta general del proyecto, y objetivos generales y específicos, que desglosan las funcionalidades clave y las capacidades esperadas del sistema. Estos objetivos servirán como la hoja de ruta para el diseño, implementación y evaluación del proyecto, asegurando que el producto final cumpla con las necesidades de los usuarios y las expectativas de la UGR en cuanto a la organización académica.

\subsection{Objetivo Principal}

Desarrollar una aplicación backend basada en microservicios robusta, escalable y segura para la gestión personalizada de horarios académicos de la Universidad de Granada (UGR), que permita a los usuarios acceder a su información horaria de manera centralizada y personalizada, facilitando la integración con servicios de calendario externos como Google Calendar para mejorar la accesibilidad y la organización.

\subsection{Objetivos Generales}

\begin{enumerate}
    \item Personalizar la visualización del horario para cada tipo de usuario según sus suscripciones.
    \item Facilitar la gestión y comunicación de cambios de horario y eventos académicos (tutorías, clases de recuperación, charlas, etc.).
    \item Permitir la integración con servicios de calendario externos para una mayor accesibilidad y sincronización de la información horaria.
\end{enumerate}

\subsection{Objetivos Específicos}

\begin{enumerate}
    \item Implementar un sistema de registro y autenticación seguro para usuarios (alumnos y profesores) utilizando correos electrónicos institucionales de la UGR.
    \item Permitir a los alumnos y profesores suscribirse y revocar suscripciones a los grupos de asignaturas de los grados que cursan / imparten.
    \item Generar y mostrar el horario personalizado de cada usuario en función de sus suscripciones, incluyendo información detallada de la asignatura, grupo, horario, profesores y aula.
    \item Permitir a los profesores y administradores crear, modificar y eliminar eventos extra a las clases oficiales ( tutorías, clases de recuperación, charlas, etc.) y notificar a los alumnos sobre estos eventos.
    \item Permitir a los usuarios exportar su horario en formato iCalendar (.ics) para su importación en diversos sistemas de calendario.
    \item Implementar la sincronización automática del horario de los usuarios con Google Calendar, reflejando los cambios en tiempo real.
\end{enumerate}
