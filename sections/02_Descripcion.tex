\chapter{Descripción del problema}\label{cap:descripcion}

La Universidad de Granada (UGR), con su amplia oferta formativa y su elevado número de usuarios —alrededor de 60 000 estudiantes y 3.500 profesores \cite{webPromocionUGR}— depende de una gestión de horarios académicos ágil y precisa para garantizar la correcta organización de sus actividades docentes. Sin embargo, en la práctica este proceso adolece de dispersión y falta de personalización: cada usuario debe rastrear múltiples portales en la web institucional para construir su propio itinerario, lo que no solo incrementa la probabilidad de errores y solapamientos, sino que también genera un coste oculto en tiempo y recursos cuya magnitud analizaremos en detalle en secciones posteriores.

\section{Contexto y problemática}

La Universidad de Granada (UGR) es una de las universidades más grandes de España, con una amplia oferta académica y un gran número de estudiantes y profesores. La gestión de horarios académicos es un aspecto crítico para el funcionamiento eficiente de la universidad. Sin embargo, la UGR podría mejorar en este campo en varios aspectos:

\begin{itemize}
    \item \textbf{Falta de personalización:} Actualmente, los estudiantes y profesores tienen acceso a un horario general que no se adapta a sus necesidades específicas. Esto dificulta la planificación y organización de su tiempo.
    \item \textbf{Dificultad en la gestión de cambios:} Los cambios en los horarios y eventos académicos (tutorías, clases de recuperación, charlas, etc.) no se comunican de manera efectiva a los estudiantes y profesores, lo que puede llevar a confusiones y malentendidos.
    \item \textbf{Integración con servicios externos:} La falta de integración con servicios de calendario externos como Google Calendar limita la accesibilidad y la organización del horario académico.
\end{itemize}

\section{Estimación del coste agregado}

Más allá de los costes económicos directos asociados a software o personal administrativo, existe un coste ``invisible'' que se manifiesta en el tiempo y el esfuerzo invertidos por los colectivos implicados en la realización de su calendario escolar personalizado. Esta sección busca cuantificar, de manera aproximada, dicho coste anual en una universidad con una población similar a la descrita anteriormente (aproximadamente 60.000 estudiantes y 3.500 profesores).
\newline\newline
Para esta estimación, consideraremos el tiempo medio que, de forma no remunerada o fuera de sus funciones explícitas, dedican estudiantes y profesores a la consulta, ajuste y resolución de problemas relacionados con la confección de horarios cada cuatrimestre. La información se ha recopilado a través de encuestas realizadas a miembros del departamento de Ingeniería de Computadores, Automática y Robótica, y a estudiantes de distintos cursos de la ETSIIT.

\subsection{Coste para el profesorado}

El profesorado dedica un tiempo considerable a la revisión de horarios, la coordinación con otros docentes, la solicitud de cambios por solapamientos o la adaptación a nuevas asignaciones. Aunque gran parte de esta labor se gestiona a través de los coordinadores de titulación y jefes de departamento, la interacción individual sigue siendo significativa.

Para obtener una estimación del tiempo invertido por el profesorado, se realizó una consulta a los profesores del departamento de Ingeniería de Computadores, Automática y Robótica (ICAR). Las respuestas obtenidas revelan una dedicación variada en la configuración y ajuste de horarios. Excluyendo las respuestas que se referían a actividades diarias o eran valores atípicos, la mayoría de las estimaciones para la configuración inicial y ajustes esporádicos se sitúan entre los 15 y 45 minutos por cuatrimestre. Para esta estimación, se ha calculado una media ponderada de las respuestas, resultando en un promedio de \textbf{30 minutos (0.50 horas) por profesor y cuatrimestre} para las tareas de configuración y ajuste de horarios.

\begin{itemize}
    \item \textbf{Población:} Aproximadamente 3.500 profesores.
    \item \textbf{Tiempo estimado por profesor y cuatrimestre:} 0.50 horas.
    \item \textbf{Coste anual por profesorado:} (3.500 profesores) $\times$ 0.50 horas/cuatrimestre $\times$ 2 cuatrimestres/año = 3.500 horas/año.
\end{itemize}

\subsection{Coste para el estudiantado}

El estudiantado es el colectivo numéricamente mayor y, por ende, el que acumula un mayor volumen de tiempo en la gestión de sus horarios. Los problemas comunes incluyen solapamientos entre asignaturas obligatorias u optativas, dificultades para encajar horarios por trabajo o estudios adicionales, y la necesidad de consultar repetidamente las plataformas hasta que los horarios se consolidan. Además los estudiantes de un Grado como puede ser el de Ineniería Informática tienen un número muy inferior de grupos por asignatura comparado por ejemplo con los estudiantes de Medicina, lo que les obliga a realizar una mayor cantidad de consultas para encontrar un horario que se ajuste a sus necesidades.\newline

Se ha realizado una encuesta anónima similar a alumnos de diferentes cursos de la ETSIIT. Las estimaciones obtenidas del estudiantado son consistentemente similares a las reportadas por el profesorado del departamento ICAR, reflejando una dedicación significativa a la consulta y gestión de sus propios horarios. En base a estas encuestas, se estima una media de \textbf{45 minutos por estudiante y cuatrimestre}. Este tiempo engloba la consulta inicial y repetida de horarios, el análisis de solapamientos, la búsqueda de soluciones y la comunicación para reportar problemas.

\begin{itemize}
    \item \textbf{Población:} Aproximadamente 60.000 estudiantes.
    \item \textbf{Tiempo estimado por estudiante y cuatrimestre:} 45 min.
    \item \textbf{Coste anual por estudiantado:} (60.000 estudiantes) $\times$ 0.75 hora/cuatrimestre $\times$ 2 cuatrimestres/año = 90.000 horas/año.
\end{itemize}

\subsection{Estimación del coste agregado anual}

Sumando el tiempo estimado para ambos colectivos, obtenemos una cuantificación de la inversión de tiempo anual en la gestión de horarios académicos en la UGR:

\begin{itemize}
    \item \textbf{Total de horas anuales:} 90.000 horas (estudiantes) + 3.500 horas (profesores) = \textbf{93.500 horas/año}.
\end{itemize}

Para contextualizar esta cifra, podemos convertirla en un equivalente de jornadas laborales o incluso en un valor económico aproximado, aunque el objetivo principal es visibilizar el volumen de tiempo.

\begin{itemize}
    \item \textbf{Equivalente en jornadas laborales (8 horas/día):} 93.500 horas/año $\div$ 8 horas/día = \textbf{9.358 jornadas laborales/año}.
\end{itemize}

Esta cifra de 93.500 horas anuales dedicadas por la comunidad universitaria a la gestión y resolución de problemas de horarios subraya la magnitud del \textbf{coste temporal} que puede pasarse por alto. Representa una cantidad significativa de tiempo y esfuerzo que podría ser redirigida hacia actividades más productivas académicamente o de mejora de la calidad de vida de la comunidad universitaria. La optimización de los procesos de generación de horarios, la mejora de la comunicación y la implementación de herramientas más eficientes tienen el potencial de mitigar sustancialmente este impacto.

\section{Solución propuesta}

Partiendo de la problemática observada, se propone la elaboración de un sistema de gestión personalizada de horarios académicos para los grados de la Universidad de Granada. 
Esto permitirá a los estudiantes y profesores acceder a su información horaria de manera centralizada y con comunicaciones efectivas sobre cambios y eventos académicos. Además, la integración con servicios de calendario externos como Google Calendar mejorará la accesibilidad y la organización del horario académico.

\section{Restricciones}

A fin de implementar la solución propuesta de manera eficaz y garantizar que responda a las necesidades detectadas en el contexto de la UGR, es imprescindible que el sistema cumpla una serie de restricciones técnicas y funcionales. Estas restricciones aseguran que la solución sea completa, segura, accesible, compatible, escalable e integrable con otros servicios, permitiendo así su adopción y correcto funcionamiento en el entorno universitario.

\begin{itemize}
    \item \textbf{Completitud:} El sistema debe ser capaz de gestionar todos los grados y asignaturas de la UGR.
    \item \textbf{Seguridad:} El sistema debe manejar la mínima información privada posible para un funcionamiento normal.
    \item \textbf{Accesibilidad:} El sistema debe ser accesible desde cualquier dispositivo con conexión a Internet.
    \item \textbf{Compatibilidad:} El sistema debe ser compatible con los navegadores web más utilizados.
    \item \textbf{Escalabilidad:} El sistema debe ser capaz de manejar un gran número de usuarios y solicitudes simultáneas.
    \item \textbf{Integración:} El sistema debe ser capaz de integrarse con servicios de calendario externos como Google Calendar.
\end{itemize}

\section{Objetivos del proyecto}

En esta sección se detallan los objetivos que guiarán el desarrollo del sistema.
Se dividen en un objetivo principal, que establece la meta general del proyecto, y objetivos generales y específicos, que desglosan las funcionalidades clave y las capacidades esperadas del sistema. Estos objetivos servirán como la hoja de ruta para el diseño, implementación y evaluación del proyecto, asegurando que el producto final cumpla con las necesidades de los usuarios y las expectativas de la UGR en cuanto a la organización académica.

\subsection{Objetivo Principal}

Desarrollar una aplicación backend basada en microservicios robusta, escalable y segura para la gestión personalizada de horarios académicos de la Universidad de Granada (UGR), que permita a los usuarios acceder a su información horaria de manera centralizada y personalizada, facilitando la integración con servicios de calendario externos como Google Calendar para mejorar la accesibilidad y la organización.

\subsection{Objetivos Generales}

\begin{enumerate}
    \item Personalizar la visualización del horario para cada tipo de usuario según sus suscripciones.
    \item Facilitar la gestión y comunicación de cambios de horario y eventos académicos (tutorías, clases de recuperación, charlas, etc.).
    \item Permitir la integración con servicios de calendario externos para una mayor accesibilidad y sincronización de la información horaria.
\end{enumerate}

\subsection{Objetivos Específicos}

\begin{enumerate}
    \item Implementar un sistema de registro y autenticación seguro para usuarios (alumnos y profesores) utilizando correos electrónicos institucionales de la UGR.
    \item Permitir a los alumnos y profesores suscribirse y revocar suscripciones a los grupos de asignaturas de los grados que cursan / imparten.
    \item Generar y mostrar el horario personalizado de cada usuario en función de sus suscripciones, incluyendo información detallada de la asignatura, grupo, horario, profesores y aula.
    \item Permitir a los profesores y administradores crear, modificar y eliminar eventos extra a las clases oficiales ( tutorías, clases de recuperación, charlas, etc.) y notificar a los alumnos sobre estos eventos.
    \item Permitir a los usuarios exportar su horario en formato iCalendar (.ics) para su importación en diversos sistemas de calendario.
    \item Implementar la sincronización automática del horario de los usuarios con Google Calendar, reflejando los cambios en tiempo real.
\end{enumerate}

\section{Licencia de código abierto}

Todos los repositorios de código fuente asociados a este proyecto están licenciados bajo la \textbf{Licencia Apache 2.0}. Esto se hace con el fin de fomentar la colaboración del estudiantado y/o profesorado, así como de la UGR y de la comunidad en general, permitiendo que cualquier persona interesada pueda contribuir al desarrollo del sistema, adaptarlo a sus necesidades o utilizarlo como base para otros proyectos.

Se ha optado por esta licencia debido a sus características y beneficios clave, que se alinean perfectamente con los objetivos de nuestro proyecto:

\begin{itemize}
    \item \textbf{Permisividad y Flexibilidad:} La Licencia Apache 2.0 permite a otros usuarios y organizaciones utilizar, modificar, distribuir y sublicenciar el software para cualquier propósito, incluyendo el uso comercial. Esto fomenta una amplia adopción y colaboración en el desarrollo del proyecto.
    \item \textbf{Claridad en la Atribución:} Si bien es permisiva, la licencia Apache 2.0 requiere que se mantengan los avisos de copyright y los términos de la licencia en todas las copias y obras derivadas. Esto asegura que el origen del proyecto sea siempre reconocido.
    \item \textbf{Protección de Patentes:} Un aspecto crucial de esta licencia es la concesión de patentes implícita. Si los colaboradores aportan código que incluye invenciones patentadas, otorgan una licencia de patente a los usuarios del software. Esto ayuda a prevenir futuros litigios por patentes y promueve un entorno de desarrollo más seguro y libre de riesgos.
    \item \textbf{Madurez y Reconocimiento:} La Licencia Apache 2.0 es ampliamente reconocida y utilizada por grandes empresas y proyectos de software de renombre mundial, lo que ofrece una base legal sólida y bien entendida por la comunidad de desarrollo.
    \item \textbf{Cambios:} La licencia obliga a los usuarios a documentar cualquier cambio realizado en el código fuente original, lo que facilita el seguimiento de las modificaciones y mejora la transparencia del proyecto.
    \item \textbf{No viral:} A diferencia de licencias como la GPL, la Apache 2.0 no impone restricciones de copyleft, lo que significa que los desarrolladores pueden incorporar el código en proyectos propietarios sin necesidad de liberar su propio código bajo la misma licencia.
\end{itemize}

Esta elección de licencia busca equilibrar la máxima libertad para la comunidad con una estructura legal clara que protege tanto a los desarrolladores originales como a los futuros usuarios del software.


