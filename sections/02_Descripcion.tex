\chapter{Descripción del problema}\label{cap:descripcion}

\section{El problema}

\section{La solución}

\section{Restricciones}

\section{Objetivos}

\subsection{Objetivo Principal}

Desarrollar una aplicación backend basada en microservicios robusta, escalable y segura para la gestión personalizada de horarios académicos de la Universidad de Granada (UGR), que permita a los usuarios acceder a su información horaria de manera centralizada y personalizada, facilitando la integración con servicios de calendario externos como Google Calendar para mejorar la accesibilidad y la organización.

\subsection{Objetivos Generales}

\begin{enumerate}
    \item Personalizar la visualización del horario para cada tipo de usuario según sus suscripciones.
    \item Facilitar la gestión y comunicación de cambios de horario y eventos académicos (tutorías, clases de recuperación, charlas, etc.).
    \item Permitir la integración con servicios de calendario externos para una mayor accesibilidad y sincronización de la información horaria.
\end{enumerate}

\subsection{Objetivos Específicos}

\begin{enumerate}
    \item Implementar un sistema de registro y autenticación seguro para usuarios (alumnos y profesores) utilizando correos electrónicos institucionales de la UGR.
    \item Permitir a los alumnos y profesores suscribirse y revocar suscripciones a los grupos de asignaturas de los grados que cursan / imparten.
    \item Generar y mostrar el horario personalizado de cada usuario en función de sus suscripciones, incluyendo información detallada de la asignatura, grupo, horario, profesores y aula.
    \item Permitir a los profesores y administradores crear, modificar y eliminar eventos extra a las clases oficiales ( tutorías, clases de recuperación, charlas, etc.) y notificar a los alumnos sobre estos eventos.
    \item Permitir a los usuarios exportar su horario en formato iCalendar (.ics) para su importación en diversos sistemas de calendario.
    \item Implementar la sincronización automática del horario de los usuarios con Google Calendar, reflejando los cambios en tiempo real.
\end{enumerate}
